\chapter{Introduction}

\lettrine{A}{tomistic simulation} is a discipline based on numerically
solving the fundamental equations of classical and quantum physics in
order to directly predict the trajectories of the atoms which make up
a physical system.
%
That this can be done in principle has been appreciated for some time
--- centuries in the case of classical mechanics --- but it is only
with the advent of computers capable of carrying out this procedure
accurately and reliably that simulation has become an indispensable
research tool, complementary both to experimental and theoretical
approaches.
%
In Chapter~\ref{chap:methods}, I review the background to two of the
main methods currently used to model physical systems at the atomic
level: classical molecular dynamics and first principles quantum
mechanical calculations.
%
The former allow simulations of millions of atoms to be carried out on
a nanosecond timescale \citep{Haile92,Rapaport04,Verlet67,Swope82},
but the accuracy is limited by the requirement to use simple
parameterisations as interatomic potentials
\citep{Lennard-Jones24,SWpotential,Tersoff_T1,
Tersoff_T2,Tersoff_T3,Brenner90,Brenner00,
Brenner02,Bazant97,Justo98,Marks00,vanDuin01,Horsfield96,Stuart00}.
%
If the scientific question of interest can be effectively answered by
considering the behaviour of a very small number of atoms, up to
around a hundred, then \emph{ab initio} approaches allow this
limitation to be overcome
\citep{Hohenberg64,Kohn65,Perdew96,Teter89,Payne92,Segal02}.
%

In many cases we can extract enough information from these accurate
quantum mechanical calculations to parametrise less transferable, but
far less expensive, models as use them on a larger length scale
\citep{Nieminen02}.
%
For some systems however, it is impossible to separate the behaviour
on the various length scales, since the coupling between them is
strong and bidirectional.
%

\section{Prova}

Then the only option is to carry out a \emph{hybrid} simulation, where
some parts of the system are treated at a higher level of accuracy
\citep{Kohlhoff91,Tadmor96,Warshel76,Maseras95}.
%
An overview of existing hybrid schemes is given in
Chapter~\ref{chap:hybrid}.

The best known example of such a multiscale system is the fracture of
brittle materials, which forms the subject of
Chapter~\ref{chap:fracture}.
%
\begin{figure}
  \begin{center}
%    \includegraphics[width=90mm]{continuum-stress.eps}
  \end{center}
  \caption[Continuum stress field near crack tip]{
    \label{fig:continuum-stress} 
    Maximum principal stress near the the tip of a crack under
    uniaxial tension in the opening mode, from the linear elastic
    solution (discussed in detail in Section
    \ref{sec:linear-elastic}). Black areas are the least stressed and
    yellow the most.}
\end{figure}
%
The conditions for crack propagation are created by stress
concentration at the crack tip, and depend on macroscopic parameters
such as the loading geometry and dimensions of the specimen
\citep{Inglis13,Griffith,Broberg99,Freund90}.
%
In real materials, however, the detailed crack propagation dynamics,
are entirely determined by atomic scale phenomena since brittle crack
tips are atomically sharp and propagate by breaking bonds, one at a
time, at each point along the crack front \citep{Lawn93,Thomson86}.
%
This means the tip region is primarily a one dimensional line,
perpendicular to the direction of propagation, and so it should be
possible to define a contiguous embedding region to be treated with a
more accurate model in a hybrid simulation.
%
There is a constant interplay between the length scales because the
opening crack gives rise to a stress field with a singularity at the
tip \citep{Irwin}, as illustrated in Fig.~\ref{fig:continuum-stress},
and in turn it is this singular stress field which breaks the bonds
that advance the crack.
%
Only by including the tens of thousands of atoms that contribute
significantly to the elastic relaxation of this stress field can we
hope to accurately model the fracture system, and thus a multiscale
approach is essential.
%
This thesis is concerned with the application of hybrid methods to the
brittle fracture of silicon, in particular I consider low-speed
fracture on the $(111)$ cleavage plane.

The major difficulty in hybrid simulation is dealing with the boundary
between the two different descriptions of material; often unphysical
forces arise here when two incompatible models are combined
\citep{Spence93,Broughton99,Bernstein01}.
%
The `Learn on the Fly' (LOTF) hybrid method
\citep{LOTF98,LOTF04,LOTF05,Moras06,DefectBook}, described in
Chapter~\ref{chap:lotf}, is a recently introduced strategy to solve
the boundary problem, at least in materials that can be described
using short range forces.
%
LOTF couples the quantum and classical dynamics in such a way that
that the system behaves instantaneously as if the entire specimen is
being treated at a quantum mechanical level of detail.
%
This is achieved by combining accurate quantum forces for the region
of interest obtained from \emph{ab initio} calculations with
qualitatively correct classical forces further away, using an
adjustable potential to give stable dynamics.

%
Chapter~\ref{chap:simulation} sets out the methodology for the
fracture simulations and describes a series of validation tests,
whilst the main results of this work are presented in
Chapter~\ref{chap:results}.
%
I find that fracture is unstable on the $(111)$ plane at low speeds;
conventionally this has been thought of as the most stable crack
plane.
%
The instability I report in this thesis is caused by a crack tip
reconstruction which triggers a positive feedback `sinking' mechanism
leading to macroscopic, experimentally observable, corrugations.
%
As discussed in Chapter~\ref{chap:discussion}, recent experiments have
observed surface features consistent with these predictions
\citep{Sherman07,Kermode07}.
%
%
For some systems however, it is impossible to separate the behaviour
on the various length scales, since the coupling between them is
strong and bidirectional.
%%
For some systems however, it is impossible to separate the behaviour
on the various length scales, since the coupling between them is
strong and bidirectional.
%%
For some systems however, it is impossible to separate the behaviour
on the various length scales, since the coupling between them is
strong and bidirectional.
%%
For some systems however, it is impossible to separate the behaviour
on the various length scales, since the coupling between them is
strong and bidirectional.
%%
For some systems however, it is impossible to separate the behaviour
on the various length scales, since the coupling between them is
strong and bidirectional.
%%
For some systems however, it is impossible to separate the behaviour
on the various length scales, since the coupling between them is
strong and bidirectional.
%%
For some systems however, it is impossible to separate the behaviour
on the various length scales, since the coupling between them is
strong and bidirectional.
%%
For some systems however, it is impossible to separate the behaviour
on the various length scales, since the coupling between them is
strong and bidirectional.
%%
For some systems however, it is impossible to separate the behaviour
on the various length scales, since the coupling between them is
strong and bidirectional.
%%
For some systems however, it is impossible to separate the behaviour
on the various length scales, since the coupling between them is
strong and bidirectional.
%%
For some systems however, it is impossible to separate the behaviour
on the various length scales, since the coupling between them is
strong and bidirectional.
%%
For some systems however, it is impossible to separate the behaviour
on the various length scales, since the coupling between them is
strong and bidirectional.
%
