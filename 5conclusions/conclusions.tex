\chapter{Conclusions} \label{chapter:concl}

\lettrine{I}{n this work} we first investigated at the molecular level the properties of the synthetic, self-assembling AMP capzip. Multiscale simulations identified a preferred geometry of assembly.
%
In particular, the simulations proved that such short sequences must assemble in a sufficiently dense way (here represented by the double layer structure) to have the structural robustness necessary to form stable three dimensional objects.
%
This concept is strictly paired with the necessity of placing residues within the structure in a way that satisfies their preferred solvent exposure.

To host both the hydrophilic and hydrophobic residues present in AMPs, very often the design of synthetic AMPs has employed natural $\alpha$-helices as templates, as they offer a geometry able to accommodate these two classes in a suitable way. The design of capzip explores the possibility of forming an AMP structure without such fold, and simulations explained how its sequence is compatible with such needs.
%
The resulting double layer capsule successfully buried the Tryoptophan residues, which are the drivers of the stability and cohesion between the layers, and expose the Arginine ones, which contribute to the arm-arm pairing.
%
In doing so, the role of specific amino acids was highlighted, and this is important in the context of designing new AMPs.

As explained, these simulations of pre-assembled structure (or even of just small subsequences) give insight on the assembly process, even if the technique employed for the investigation, Molecular Dynamics, is yet quite limited to assess it at a coarse detailed level, and impossible at all for a finer (atomistic) level.

The outlook and future effort of simulations on capzip assembly we would like to pursue is the exploration of different amino acids composition for the capzip arms. This can hardly be done in a systematic manner even at the MARTINI coarse grain level. Thus, ultra coarse grain models where capzip is represented with three/four beads for each arms can be employed for a first screen of which electrostatic and hydrophobic property each segment should have to tailor the assembly. Preliminary work has been done to set up such parametrisation, being informed by the results of the atomistic and coarse grain simulations (e.g.\ on the flexibility of the block, and the interaction between hydrophobic patches).

The second contribution of this work was the assessment of the antimicrobial activity of capzip on model membranes. In such simplified framework, we assessed that capzip interacts more favourably with a membrane composed by mixed zwitterionic and anionic phospholipids (which represents the bacterial inner membrane), but less with a membrane composed only by zwitterionic ones (representing the mammalian membrane). Additionally, the peptide decreases the stability of the membrane and its ability to withstand external perturbation like the one of an electric field.

The disruptive action can be compare with many well know cases of Arginine insertion.  Instead, Tryptophan (hydrophobic) residues do not seem to penetrate deep into the membrane instead.
%
This is in disagreement with the picture generally referred to as standard, which proposes a two step process of charge binding and hydrophobic insertion. However, this process was designed with helical AMPs as reference, and we proved that for a self-assembly of AMP with high $\beta$-sheet content, the Arginine insertion (under electric field conditions) was sufficient to trigger poration without rearrangement of all the Tryptophan in the membrane. The fact that many capzip molecules are present in the same location is crucial in preventing the pore sealing, showing the importance of an assembled structure.

This investigation necessarily explores specific, simplified membranes. The results correlate with the experiments on Supported Lipid Bilayer of the same composition, informing the mechanism that allow their disruption. However, they must be reconciled with the complex structure of both bacterial and mammalian membranes: the bacterial model used is resemblant of the inner membrane, so that the mechanism of permeation and/or transportation of the peptide through the outer membrane (for Gram-negative bacteria) or the peptidoglycan layer (for Gram-positive ones) remains still to be elucidated.
%
In this context once more we foresee that the assembly property plays an important role: the assembled capsule is a highly positively charged object while the overall bacterial cell wall is negatively charged, so the attraction effect observed in simulations with a model anionic membrane can be extrapolated also for this case. 

On the contrary, the scarce interaction with the bacterial membrane poses the question of which determinants are then favouring a good internalisation of siRNA molecules in HeLa cells: if simulations prove a low affinity of the capsule with the membrane, and experiments prove a low affinity of siRNA alone, the internalisation observed when the peptide and the siRNA are combined is likely favoured by the balance of positive and negative charges of the two combined together. This influences the outlook of capzip applications for therapy delivery, as they must be focussed on the delivery of negatively charged molecule.
%
Indeed the capsule itself has the potential to transport a broader variety of drugs, as proved by preliminary coarse grain simulations of capzip capsule in conjunction with doxorubicin (a hydrophobic compound), however, this has little application on mammalian cell if the resulting object, positively charged, can not deliver its content.

This considerations makes an exciting ground for boosting the application of capzip not only as an antimicrobial agent but also as a delivery agent, fully exploiting the multi-functional behaviour it was designed for.


Two other contributions of this work concern methodological aspects, i.e.\ the evaluation and improvement of Molecular Dynamic simulations.

First, connected with the exploration of capzip assembly, we performed a systematic analysis of the performances of different coarse grain force field in simulating the assembly of the short peptide capzip. In particular, while analysing the different structures they produced, we proved that the SIRAH force field returns an energetic profile closer to the atomistic one, while the MARTINI models proposed a different balance between the electrostatic and short range components. Despite these conclusions could be guessed from the parametrisation strategy, to our knowledge this is the first comparison between SIRAH and MARTINI.
This is important for the choice of the coarse grain model, in particular when simulating systems with a high net charge, as the relative weight of the Coulomb interactions versus the others must be considered when comparing the results against atomistic ones.

Finally, we carefully analysed and questioned the parametrisation of lipids in the GROMOS force field, proposing a more up to date version of parameters for phopsholipids and phosphocholine in general (version 54A8\_v1, available at \url{http://fraternalilab.kcl.ac.uk/wordpress/biomembrane-simulations/}). This novel parametrisation has a significant impact on the simulations of proteins-lipids interaction, as proved by test simulations of the AMP Lfcin. We think that this parameters are a first step toward reconciling the description of the two components, to model their mutual influence at best.
%
This work places itself in the the continuous improvement in MD simulations parameters, which is constantly prompted by three factors: the better experimental results available, the development of new strategies of parametrisation, and the evolution of computers, which allows the simulations of larger systems and longer time scales, verifying or questioning what already simulated in reduced systems or shorter runs.
%
The past successes of MD simulations already show the potentiality of this techniques which nevertheless must be kept updated and performing in face of new challenges.
As such, we are keen in testing more extensively the new parameters on other systems and to repeat selected simulations of capzip on model membranes to understand the type of interactions that they propose.

Overall, this work elucidate assembly and antimicrobial mechanism of a synthetic molecule by use of computational techniques, proving that they are an irreplaceable tool to access the atomistic details of such processes.

