\chapter{Conclusions} \label{chapter:concl}

\lettrine{I}{n this work} we sought to understand capzip molecule-molecule interactions, to characterise at the molecular level the structures this antimicrobial peptide forms in solution. In particular, we wanted to identify which amino acids or sub-structures play a key role in pairing the molecules. 
%
Multiscale simulations identified a network of $\beta$-sheets organised in a double layer as the preferred geometry of assembly. The presence of a second layer was proved necessary to form stable three dimensional objects.

This need is strictly paired with the necessity of organising residues in a way that satisfies their preferred solvent exposure.
%
To host both the hydrophilic and hydrophobic moieties present in AMPs, rigid $\alpha$-helical structures have been widely employed in the past as they can accommodate both of them in a suitable way. However, the design of capzip explores the possibility of forming AMP structures without such fold, and simulations explained how the unstructured sequence becomes compatible with an amphiphatic structure once many copies are assembled together.
%
The resulting double layer capsule successfully buries Tryptophan residues, which are the drivers of the stability and cohesion between the layers, and expose Arginine ones, which contribute to the arm-arm pairing.
%
To pinpoint the role of specific amino acids was one of the aim of the work and is important in the context of designing new AMPs, as indeed a possible future direction of capzip investigation is the exploration of different amino acids composition for the molecule arms. This can hardly be done in a systematic manner both at the experimental level and \emph{in silico}, even employing coarse-grained simulations. Thus, ultra coarse-grained models can be employed in which capzip is represented with three/four beads for each arms. This would provide a first screen of which electrostatic and hydrophobic property each segment should have to tailor the assembly.

The question which remains open in the investigation of the assembly concerns the kinetics of the process. However, the simulations of pre-assembled structures gave already insight on it, while the full event can likely be monitored only at the same ultra coarse-grained level useful for the exploration of mutated sequences.

The second question this work aimed to answer was how the antimicrobial activity of capzip was performed, understanding which moieties actively perturbed the membrane and how such perturbation could be quantified. Choosing the simplified framework of model membranes, we assessed that capzip interacts favourably with a membrane composed by mixed zwitterionic and anionic phospholipids (which represents the bacterial inner membrane), but less with a membrane composed only by zwitterionic ones (representing the mammalian membrane). The peptide decreases the stability of the membrane and its ability to withstand external perturbation like the one of an electric field. This explain how capzip promotes poration on bacterial membranes only.

The disruptive action is promoted by charged Arginine insertion, while hydrophobic Tryptophan residues do not seem to penetrate deeply into the membrane.
%
This is in disagreement with the picture generally referred to as standard, which proposes a two step process of charge binding and hydrophobic insertion. However, this process was designed with helical AMPs as reference, and here we proved instead that for an assembly of AMP with high $\beta$-sheet content, the Arginine insertion (under electric field conditions) is sufficient to trigger poration without rearrangement of all the Tryptophan within the membrane.
%
The assembly of capzip is crucial in bringing many antimicrobial sequences at the same position on the membrane, creating a localised high positive charge. Moreover, through the presence of many Arginine residues, water molecules inserting into the hydrophobic membrane core on a casual basis are more easily converted into stable pores.

This investigation necessarily explores specific, simplified membranes. The results correlate with the experiments on Supported Lipid Bilayer of the same composition, informing the mechanism that allow their disruption. However, they must be reconciled with the complex structure of both bacterial and mammalian real membranes: the bacterial model used is mimicking the inner membrane, and therefore the mechanism of permeation and/or transportation of the peptide through the outer membrane (for Gram-negative bacteria) or the peptidoglycan layer (for Gram-positive ones) remains still to be elucidated.
%
However, the bacterial cell wall is negatively charged, as our model of bacterial membrane. Therefore we can assume that the mechanism of adsorption observed in our simulations is in broad sense reproducing the physiological situation.
%
In this context once more we foresee that the assembly property plays an important role: the assembled capsule is a highly positively charged object while the overall bacterial cell wall is negatively charged, so the attraction effect observed in simulations with a model anionic membrane will be valid also for this case, despite the details of the membrane compositions are different. 
%The open question is therefore on the selectivity of this AMP against bacteria based on the detailed composition of their membranes, bearing in mind that capzip has been proven effective against a broad range of bacteria, suggesting a non specific mechanism, compatible with our findings which show the antimicrobial action taking place on commonly present phospholipids.

On the contrary, the scarce interaction with the mammalian membrane poses the question of which determinants are then favouring a good internalisation of siRNA molecules in HeLa cells, as proven by experiments. This system is very different with respect to the pure capzip assembly because of the negative charge of siRNA. As such, it has not been investigated yet. Its analysis is within the future outlook of this work, which is important foreseeing applications of capzip for delivery of alternative therapies.
%
Indeed, the capsule itself has the potential to transport a broad variety of drugs, rather than siRNA only, but the characteristics of the capzip-drug assembly must be uncovered in each case to understand their interactions with bacterial and mammalian membranes.
%
These considerations make an exciting ground for boosting the application of capzip not only as an antimicrobial agent but also as a delivery agent, fully exploiting the multi-functional behaviour it was designed for.

Finally, this work brought two contributions to methodological aspects of simulations techniques: they emerged as a valuable by-product of the \emph{in silico} exploration chosen, in the effort of developing and applying the best and more updated techniques to our system.

First, connected with the exploration of capzip assembly, we performed a systematic analysis of the performances of different coarse-grained force fields (SIRAH, MARTINI, Polar MARTINI) in simulating the assembly of the capzip. In particular we proved that the SIRAH force field returns an energetic profile closer to the atomistic one, while the MARTINI models proposed a remarkably different balance between the electrostatic and short range energetic components. These conclusions could be guessed from the respective parametrisations, as SIRAH groups fewer atoms than MARTINI in one bead, however to our knowledge this is the first comparison which present a quantitative analysis of it.
%
This is important for the choice of the coarse-grained model, in particular when simulating systems with a high net charge, as the relative weight of the Coulomb interactions versus short range ones may vary considerably.

Finally, we carefully analysed and questioned the parametrisation of lipids in the GROMOS force field, proposing a more up to date version of parameters for phopsholipids and phosphocholine in general (called version 54A8\_v1, available at \url{http://fraternalilab.kcl.ac.uk/wordpress/biomembrane-simulations/}). This novel parametrisation has a significant impact on the simulations of proteins-lipids interaction, as proved by test simulations of the AMP Lfcin. We think that this parameters are a first step toward reconciling the description of the two components, to model their mutual influence at best.
%
This work places itself in the workflow of continuous improvements of MD simulations parameters, which is constantly prompted by three factors: better experimental results available, the development of new strategies of parametrisation, and the evolution of computers, which allows the simulations of larger systems and longer time scales, verifying or questioning what already achieved exploring reduced systems or shorter times.
%
The past successes of MD simulations already show the potential of this techniques which nevertheless must be continuously improved to tackle more complex systems.
As such, we are keen in testing more extensively the new parameters and to repeat selected simulations of capzip on model membranes to understand the type of interactions that they propose.

Overall, this work elucidate some principles and rules in the assembly and antimicrobial mechanism of actions of a synthetic molecule by use of computational techniques: these processes are inaccessible to experiments, proving that Molecular Simulations are an irreplaceable tool to access their atomistic details.

