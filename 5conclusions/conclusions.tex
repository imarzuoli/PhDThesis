\chapter{Conclusions} \label{chapter:concl}

\lettrine{I}{n this work} we sought to understand and characterise at the molecular level the structures of a nanocapsule formed by antimicrobial peptides AMPs (named capzip) in solution. This assembly could not be determined by conventional structural investigations. In particular, we wanted to identify which amino acids or sub-structures play a role in the assembly to a nanocapsule unit and in its antimicrobial function.
%
Multiscale simulations identified a network of $\beta$-sheets organised in a double layer as the preferred geometry for the assembly. The presence of a second layer was proved necessary to form a stable three dimensional object.
%
This requirement allows to optimise the propensity of the residues in the sequence for solvent exposure by burying hydrophobic residues and exposing charged residues necessary to the antimicrobial activity.

To host both the hydrophilic and hydrophobic moieties present in AMPs, rigid $\alpha$-helical structures have been in the past widely employed as they can accommodate both of them in a suitable way. However, the design of the peptide capzip explored the possibility of forming AMP structures without such fold. Our simulations have explained how the assembly of capzip is compatible with an amphiphatic structure formed by $\beta$-sheet peptides, once many copies are assembled together in a specific manner.
%
The resulting double layer capsule successfully buries Tryptophan residues, which are the drivers of the stability and cohesion between the layers, and expose Arginine ones, which contribute to the $\beta$-sheet pairing.
%
To pinpoint the role of specific amino acids was one of the aim of the work and it is particularly important in the context of designing new AMPs, as indeed a possible future direction of this work is the exploration of different amino acids composition for the molecule. This can hardly be done in a systematic manner both at the experimental level and \emph{in silico}, even employing coarse-grained simulations. Thus, ultra coarse-grained models could be employed in which capzip is represented with three/four beads for each strand. This would provide an important screening to understand which electrostatic and hydrophobic properties each segment should have to tailor the assembly.

The simulations of pre-assembled structures gave insight into the driving forces to assembly stability, particularly in comparing different coarse-grained force fields. One question which remains open concerns details on the kinetics of the process.

The second question this work aimed to answer was how the antimicrobial activity of the nanocapsule was performed, understanding which moieties actively perturbed the membrane and how such perturbation could be quantified. We used model membranes matching different experimental compositions, namely a bacterial membrane composed by zwitterionic and anionic phospholipids, and a mammalian one composed only by zwitterionic phospholipids.
%
We assessed that capzip interacts favourably with the bacterial membrane, but less favourably with the mammalian one. This property is needed to selectively exploit the antimicrobial activity.
%
The peptide decreases the stability of the bacterial membrane and its ability to withstand external perturbation like the one of an electric field. This observations illustrate how capzip can promote poration of the bacterial membranes only.

The disruptive action is lead by charged Arginine insertion in the hydrophobic membrane region, while we do not observe the hydrophobic residues (Tryptophan) to penetrate deeply into it.
%
This is in disagreement with the generally accepted picture of antimicrobial peptide insertion into membranes, which proposes a two step process formed by a charge binding event followed by hydrophobic insertion. However, this mode of action was postulated for helical AMPs. Here we proved instead that for an assembly of AMPs with high $\beta$-sheet content, the Arginine insertion is sufficient to trigger poration without rearrangement of all the Tryptophan within the membrane by applying an electric field close to physiological values.
%
The assembly of capzip is crucial in bringing many antimicrobial sequences within the same region of the membrane, creating \emph{loci} with high positive charge. Moreover, through the presence of many Arginine residues, a chain of water molecules inserting into the hydrophobic membrane core triggers the formation of a pore, that leads to the disruption of the membrane.

This investigation necessarily explores specific, simplified membranes. The results correlate with the experiments on Supported Lipid Bilayer of the same composition, informing the mechanism that allow their disruption. However, they must be reconciled with the complex structure of both bacterial and mammalian real membranes: the bacterial model used is mimicking the inner membrane, and therefore the mechanism of permeation and/or transportation of the peptide through the outer membrane (for Gram-negative bacteria) or the peptidoglycan layer (for Gram-positive ones) remains still to be elucidated.
%
However, the bacterial cell wall is negatively charged, as in the adopted model. Therefore we can assume that the mechanism of adsorption and disruption observed in our simulations is in broad sense reproducing the physiological situation.
%
%The open question is therefore on the selectivity of this AMP against bacteria based on the detailed composition of their membranes, bearing in mind that capzip has been proven effective against a broad range of bacteria, suggesting a non specific mechanism, compatible with our findings which show the antimicrobial action taking place on commonly present phospholipids.

On the contrary, the scarce interaction observed with the mammalian membrane poses the question of which determinants are then favouring a good internalisation of siRNA molecules in HeLa cells, as proven by experiments \cite{Castelletto2016}. The resulting system (siRNA plus capzip assembly) is very different with respect to the pure capzip assembly because of the negative charge contributed by siRNA molecules. As such, it has not been investigated in this work. The modelling of this is within the future outlook of this work, important for applications of capzip for delivery and alternative therapies.
%
Indeed, the capsule itself has the potential to transport a broad variety of drugs, rather than siRNA only, but the characteristics of the capzip-drug assembly must be specifically modelled for each case to understand the interaction of the nanocapsule carrier with bacterial and mammalian membranes.
%
These considerations make an exciting ground for boosting the application of capzip assembly not only as an antimicrobial agent but also as a delivery agent, fully exploiting the multi-functional behaviour it was designed for.

Finally, this work brought two contributions to methodological aspects of simulations techniques: they emerged as a valuable by-product of the \emph{in silico} exploration chosen, in the effort of developing and applying the best and more updated techniques to our system.

First, connected with the exploration of capzip assembly, we performed a systematic analysis of the performances of different coarse-grained force fields (SIRAH, MARTINI, Polar MARTINI) in simulating the assembly of the capzip. In particular we proved that the SIRAH force field returns an energetic profile closer to the atomistic one, while the MARTINI models proposed a remarkably different balance between the electrostatic and short range energetic components. These conclusions could be a result of the model underlying the respective parametrisations, as SIRAH groups fewer atoms than MARTINI in one bead, however, to our knowledge, this is the first comparison which presents a quantitative analysis of it.
%
This is important for the choice of the coarse-grained model, in particular when simulating systems with a high net charge, as the relative weight of the Coulomb interactions versus short range ones may vary considerably.

Finally, we carefully analysed and questioned the parametrisation of lipids in the GROMOS force field, proposing a more up to date version of parameters for phopsholipids and phosphocholine in general (called version 54A8\_v1, available at \url{http://fraternalilab.kcl.ac.uk/wordpress/biomembrane-simulations/}). This novel parametrisation has a significant impact on the simulations of proteins-lipids interaction, as proved by test simulations of the AMP Lfcin. We think that this parameters are a first step toward reconciling the description of the two components, to model their mutual influence at best.
%
This work places itself in the workflow of continuous improvements of MD simulations parameters, which is constantly prompted by three factors: better experimental results available, the development of new strategies of parametrisation, and the evolution of computers, which allows the simulations of larger systems and longer time scales, verifying or questioning what already achieved exploring reduced systems or shorter times.
%
The past successes of MD simulations already show the potential of this techniques which nevertheless must be continuously improved to tackle more complex systems.
As such, we are keen in testing more extensively the new parameters and to repeat selected simulations of capzip on model membranes to understand the type of interactions that they propose.

Overall, this work elucidate some principles and rules in the assembly and antimicrobial mechanism of actions of a synthetic molecule by use of computational techniques: these processes are inaccessible to experiments, proving that Molecular Simulations are an irreplaceable tool to access their atomistic details.

