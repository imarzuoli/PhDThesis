% Title Page 
%\thispagestyle{empty}

%\begin{center}
%\Large {\textsc{Multiscale Hybrid Simulation\\ of Brittle Fracture}}
%\end{center}

%\cleardoublepage

\thispagestyle{empty}

\begin{center}

{\textsc {\Huge Gaussian process models}}\\
%
\vspace{.15in}
%
{\textsc {\Huge for force fields and }}\\
%
\vspace{.15in}
%
{\textsc {\Huge wave functions}}
%
\vspace{1.5in}

{\Large Aldo Glielmo}

\vspace{.5in}

{\textsc {\large Department of Physics}}

\vspace{0.15in}

{\textsc {\large King's College London}}

%\vspace{1.7in} original
\vspace{1.3in}
%\includegraphics[height=3cm]{pem.eps}
%\hspace{1.5cm}
\includegraphics[height=3cm]{KCL_logo.pdf}

\vspace{1in}

{\textsc {\large This dissertation is submitted for \\ 
the degree of Doctor of Philosophy}}

\vspace{0.5in}

{\textsc {\large January 2019}}

\end{center}

\cleardoublepage

%\onehalfspacing

\makeatletter
\renewcommand{\@pnumwidth}{2em}
\renewcommand{\@tocrmarg}{3em}
\setlength{\cftbeforechapterskip}{.9em}
\makeatother

%%%%%%%%%%%%%%%%%%%%%%%%%%%%%%%%%%%%%%%%%%%%%%%%%%

\thispagestyle{empty}

\begin{vplace}[0.15]

\begin{center}
%\large
\it{To my parents, who provided me with the tools that make me a human being, and to the memory of Sandro, who showed me his way of using them.}
\end{center} 

\end{vplace}

%\begin{center}
%%\large
%\it{Always act like there's endless time} 
%\end{center} 


%%%%%%%%%%%%%%%%%%%%%%%%%%%%%%%%%%%%%%%%%%%%%%%%%%

\cleardoublepage
%%% empty page
\newpage
\thispagestyle{plain} % empty
\mbox{}
%%

%%%%%%%%%%%%%%%%%%%%%%%%%%%%%%%%%%%%%%%%%%%%%%%%%%

\thispagestyle{empty}
\chapter*{Declaration}

%\section*{Declaration of Originality} 

\noindent This dissertation describes work I have carried out between October 2015 and December 2018 at the department of physics of King's College London, under the supervision of Professor Alessandro De Vita (first supervisor from October 2015 to October 2018), Doctor George Booth (first supervisor from October 2015) and Professor Peter Sollich (second supervisor).

\vspace{.2cm}

\noindent This dissertation contains material appearing in the following articles:

\begin{itemize}
%\end{itemize}}[(i)]
\item A. Glielmo, C. Zeni, \'A. Fekete and A. De Vita. Building nonparametric $n$-body force fields using Gaussian process regression. Submitted to K. T. Sch\"utt, S. Chmiela, A. von Lilienfeld, A. Tkatchenko, K. Tsuda, and K. R. M\"uller, editors, \emph{Machine learning for quantum simulations of molecules and materials} (Springer),
%
\item A. Glielmo, C. Zeni and A. De Vita. Efficient nonparametric $n$-body force fields from machine learning. \emph{Physical Review B}, 97, 2018,
%
\item A. Glielmo, P. Sollich and A. De Vita. Accurate interatomic force fields via machine learning with covariant kernels. \emph{Physical Review B}, 95, 2017.
%
\end{itemize}

\noindent In addition to the above, I have contributed to the following publications during the course of my PhD:

\begin{itemize}
%
\item M. Cucuringu, P. Davies, A. Glielmo, H. Tyagi. SPONGE: A generalized eigenproblem for clustering signed networks. Submitted to \emph{International Conference on Artificial Intelligence and Statistics 22},
%
\item F. Bianchini, A. Glielmo, J. R. Kermode, A. De Vita. Enabling QM-accurate simulation of dislocation motion in  gamma-Ni and  alpha-Fe using a hybrid multiscale approach. Submitted to \emph{Physical Review Materials},
%

\item N. Gunkelmann, D. Serero, A. Glielmo, M. Montaine, M. Heckel, T. P\"oschel. Stochastic nature of particle collisions and its impact on granular material properties. Submitted to S. Antonyuk, editor, \emph{Particles in contact} (Springer),
%

\item C. Zeni, K. Rossi, A. Glielmo, \'A. Fekete, N. Gaston, F. Baletto and A. De Vita. Building machine learning force fields for nanoclusters. \emph{The Journal of Chemical Physics}, 148, 2018,

\item M. Heckel, A. Glielmo, N. Gunkelmann, T. P\"oschel. Can we obtain the coefficient of restitution from the sound of a bouncing ball?. \emph{Physical Review E}, 93, 2016.
%
\end{itemize}
%


%\section*{Statement of Length}

\vspace{.2cm}

\noindent This dissertation is my own work and contains nothing which is the outcome of work done in collaboration with others, except as specified in the text and acknowledgements.
%
It has not been submitted in whole or in part for any degree or diploma at this or any other university.

\vspace{1.5cm}

\begin{raggedleft}
Aldo Glielmo \\
January 2019

\end{raggedleft}


%%%%%%%%%%%%%%%%%%%%%%%%%%%%%%%%%%%%%%%%%%%%%%%%%%%
%
%\cleardoublepage
%%%% empty page
%\newpage
%\thispagestyle{plain} % empty
%\mbox{}
%%%
%
%%%%%%%%%%%%%%%%%%%%%%%%%%%%%%%%%%%%%%%%%%%%%%%%%%%

\thispagestyle{empty}
\chapter*{Acknowledgements}
%
...


%%%%%%%%%%%%%%%%%%%%%%%%%%%%%%%%%%%%%%%%%%%%%%%%%%%
%
%\cleardoublepage
%%%% empty page
%\newpage
%\thispagestyle{plain} % empty
%\mbox{}
%%%
%
%%%%%%%%%%%%%%%%%%%%%%%%%%%%%%%%%%%%%%%%%%%%%%%%%%%

\thispagestyle{empty}
\chapter*{Summary}

\begin{center}

{\Large\sffamily Gaussian process models for force fields and wave functions}

\vspace{.2cm}

{\large\sffamily Aldo Glielmo}

\vspace{.03cm}

{\large\sffamily King's College London}
%
\vspace{.2cm}

\end{center}
%
\noindent
%
Algorithms capable of extracting information from data are increasingly finding application in condensed matter physics.
%
...
%%%%%%%%%%%%%%%%%%%%%%%%%%%%%%%%%%%%%%%%%%%%%%%%%%

\cleardoublepage
%%% empty page
\newpage
\thispagestyle{plain} % empty
\mbox{}
%%

%%%%%%%%%%%%%%%%%%%%%%%%%%%%%%%%%%%%%%%%%%%%%%%%%%
 
\tableofcontents*

\cleardoublepage

\listoffigures*

%%% List of symbols and acronyms %%%

% name
\renewcommand{\nomname}{List of symbols and acronyms}

% spacing 
\newlength{\nomitemorigsep}
\setlength{\nomitemorigsep}{\nomitemsep}
\setlength{\nomitemsep}{-\parsep}

% groups
\renewcommand{\nomgroup}[1]{
%
\itemsep\nomitemorigsep
\ifthenelse{\equal{#1}{A}}{\item[\textbf{Acronyms}]}{
%
\ifthenelse{\equal{#1}{B}}{\item[\textbf{Math and Gaussian process conventions}]}{
%
\ifthenelse{\equal{#1}{K}}{\item[\textbf{Kernels for force field learning}]}{
%
\ifthenelse{\equal{#1}{W}}{\item[\textbf{Wave function learning}]}{
%
\ifthenelse{\equal{#1}{G}}{\item[\textbf{Gaussian processes}]}{
%
\ifthenelse{\equal{#1}{F}}{\item[\textbf{Force field learning}]}{
%
\ifthenelse{\equal{#1}{KW}}{\item[\textbf{Kernels for wave function learning}]}{
%
}}}}}}}
\itemsep\nomitemsep}


\printnomenclature[7.6em]

\cleardoublepage
%%% empty page
\newpage
\thispagestyle{plain} % empty
\mbox{}
%%


\thispagestyle{empty}





