% Title Page 
%\thispagestyle{empty}

%\begin{center}
%\Large {\textsc{Multiscale Hybrid Simulation\\ of Brittle Fracture}}
%\end{center}

%\cleardoublepage

\thispagestyle{empty}

\begin{center}


{\textsc {\Huge Elucidating self-assembly}}\\
%
\vspace{.15in}
%
{\textsc {\Huge and antimicrobial strategies}}\\
%
\vspace{.15in}
%
{\textsc {\Huge of synthetic peptides:}}\\
%
\vspace{.15in}
%
{\textsc {\Huge an in silico investigation}}
%
\vspace{1.5in}

{\Large Irene Marzuoli}

\vspace{.4in}

{\textsc {\large Randall Centre for Cell and molecular Biophysics}}

\vspace{0.15in}

{\textsc {\large King's College London}}

%\vspace{1.7in} original
\vspace{1in}
%\includegraphics[height=3cm]{pem.eps}
%\hspace{1.5cm}
\includegraphics[height=3cm]{KCL_logo.pdf}

\vspace{1in}

{\textsc {\large This dissertation is submitted for \\ 
the degree of Doctor of Philosophy}}

\vspace{0.5in}

{\textsc {\large October 2019}}

\end{center}

\cleardoublepage

%\onehalfspacing

\makeatletter
\renewcommand{\@pnumwidth}{2em}
\renewcommand{\@tocrmarg}{3em}
\setlength{\cftbeforechapterskip}{.9em}
\makeatother

%%%%%%%%%%%%%%%%%%%%%%%%%%%%%%%%%%%%%%%%%%%%%%%%%%

\thispagestyle{empty}

\begin{vplace}[0.15]

\begin{center}
%\large
\it{To my parents, for letting me go}
\end{center} 

\end{vplace}


%%%%%%%%%%%%%%%%%%%%%%%%%%%%%%%%%%%%%%%%%%%%%%%%%%

%\cleardoublepage
%%% empty page
%\newpage
%\thispagestyle{plain} % empty
%\mbox{}
%%

%%%%%%%%%%%%%%%%%%%%%%%%%%%%%%%%%%%%%%%%%%%%%%%%%%

%\thispagestyle{empty}
\chapter*{Declaration}
\begin{onehalfspacing}

\noindent This dissertation describes work I have carried out between October 2016 and September 2019 at the Randall Centre for Cell and Molecular Biophysics of King's College London, under the supervision of Prof.\ Franca Fraternali (first supervisor) and Dr.\ Chris D. Lorenz (second supervisor, Department of Physics).

\vspace{.2cm}

\noindent This dissertation contains material appearing in the following articles:

\begin{itemize}
%\end{itemize}}[(i)]
\item Kepiro, I. E., Marzuoli, I., Hammond, K., et al. (submitted and in second revision). Engineering chirally blind protein pseudo-capsids into nanoprecise antibacterial persisters. \\
\emph{Chapter \ref{chapter:capzip_results} contains results presented in this paper. Appendix \ref{app:paper} contains the submitted version.}
\item Marzuoli, I., Margreitter, C., Fraternali, F. (2019). Lipid Head Group Parameterization for GROMOS 54A8: A Consistent Approach with Protein Force Field Description. \emph{Journal of Chemical Theory and Computation}, 15(10):5175--5193\\
\emph{This paper is presented in Chapter \ref{chapter:lip_par}.}
\end{itemize}

\noindent In addition to the above, I have contributed to the following publication during the course of my PhD:

\begin{itemize}
%
\item Milano, G., Marzuoli, I., Lorenz, C. D., and Fraternali, F. (2017). Self-assembly at the multi-scale level: challenges and new avenues for inspired synthetic biology modelling. In Ryadnov, M., Brunsveld, L., and Suga, H., editors, \emph{Synthetic Biology: Volume 2}, pages 35-64. The Royal Society of Chemistry, London.
%
\end{itemize}

%\section*{Statement of Length}

\vspace{.2cm}

\noindent This dissertation is my own work and contains nothing which is the outcome of work done in collaboration with others, except as specified in the text and acknowledgements.
%
It has not been submitted in whole or in part for any degree or diploma at this or any other university.

\vspace{1.5cm}

\begin{raggedleft}
Irene Marzuoli \\
October 2019

\end{raggedleft}
\end{onehalfspacing}


%%%%%%%%%%%%%%%%%%%%%%%%%%%%%%%%%%%%%%%%%%%%%%%%%%%
%
%\cleardoublepage
%%%% empty page
%\newpage
%\thispagestyle{plain} % empty
%\mbox{}
%%%
%
%%%%%%%%%%%%%%%%%%%%%%%%%%%%%%%%%%%%%%%%%%%%%%%%%%%

\thispagestyle{empty}
\chapter*{Acknowledgements}
%
\begin{onehalfspacing}
Many people contributed to my scientific and personal growth throughout my PhD, and this thesis would not have been possible without them.

First of all, Franca, an excellent scientist, supervisor and woman: she showed me how true enthusiasm for research and life can withstand thousands everyday difficulties.

Then, all the present and past members of the Fraternali's group: Anna, Carlos, Christian, Emma, Jamie, Joseph, Marius and Sun, each of you made the office a more enjoyable and scientifically stimulating place. A special thanks to Carlos and Christian, for the great (and sometimes unexpected) collaborations; and to the people who shared these years at the same pace: Joseph, whose cheerfulness made working days and scientific discussions more pleasant (not to mention the treats he supplied), and Marius, for the curious and caring attitude which granted rewarding, serendipitous, conversations (and lunches in the sun).

I am also grateful to my second supervisor, Chris, and all his group, for the concrete and prompt support on practical - or less practical - problems, whenever I needed it.

Thanks to Prof. Giuseppe Milano for hosting me and sharing his knowledge, and to Dr. Jens Kleinjung for help and suggestions in the initial stages of my project.

To all the members of the CANES community, students and academics, thank you for pushing me to stay updated on all aspects of science. In particular, I want to single out the ``italian subset", for the warmth, the support, and the encounters at all the corners of the world. Especially, thanks to Claudio, for his presence in good and difficult moments, and all the jazz music; to Aldo, for challenging (and supporting) my ideas about research, universe and everything; and mostly to Carla, a colleague, a flatmate, a friend: navigating our job and every day life together made me a more serene person.

I owe my perseverance also to Laura, Celeste and Nicola, who, comparing their PhD experience with mine, formed an international, and interdisciplinary, support network.

Finally, my fondest gratitude goes to my parents. For encouraging and trusting me in this adventure, without asking back anything, a part from brief stories from my research life - those were the only presentations I truly enjoyed giving.

\end{onehalfspacing}


\clearpage
\thispagestyle{empty}
\chapter*{Abstract}

%\begin{center}
%{\Large\sffamily \textbf{Abstract}}
%\vspace{1cm}
%\end{center}

\begin{onehalfspacing}

\noindent
%
The modern insurgence of antimicrobial resistance prompted the research of new drug alternatives. In parallel, the problem of their delivery has stimulated the research of novel biomimietic vehicles.
%
Synthetic materials can be designed to perform both functions effectively. Recently engineered nanocapsules were shown to promote bacterial membrane poration and gene delivery into mammalian cells. Their constitutive molecule, capzip, is a three branched peptide, which contains sequences inspired from a naturally occurring antimicrobial peptide (AMP). AMPs act on bacteria disrupting their membrane, a mechanism which does not strongly promote resistance.

As the atomistic details of the capzip nanocapsule assembly are still unknown, this project studied its structure in water and its interaction with model membranes, by means of multiscale Molecular Dynamics simulations. 
%
The \emph{in silico} investigation clarified the preferred structure that capzip adopts in order to form robust capsules.
In particular, the original formulation of capzip included an amphiphilic pattern to promote antimicrobial activity, but simulations proved that this has a key role in granting structural stability as well. This provides insight for the development of the next generation of multi-branched antimicrobial molecules.
%
The multiscale investigation performed prompted also a comparison between coarse-grained force fields, which will contribute to inform the choice of the most adapt one for future simulations of large peptidic assemblies.

The structures found to be stable in solution were selected for further simulations at the interface with a model bacterial and mammalian membrane.
%
The insertion of charged residues in the membrane ester region produced a local decrease in lipid mobility and, under the effect of an externally applied electric field in the physiological range, pore formation with subsequent membrane disruption.
%
Coarse-grained simulations confirmed these findings, clarifying the attraction mechanism between the capsule and the model bacterial membrane. Moreover, they suggested a lower affinity with the model mammalian membrane.
%
The exploration of the peptide-membrane interactions prompted an investigation of the currently used lipid parameters in the GROMOS atomistic force field. Given the inconsistency between the parametrisation of proteins and lipids found in the latest versions of the force field, we proposed a new parametrisation that reconcile these. The new parameters showed a peptide-membrane interaction which is less biased by the simulation's initial conditions.

\end{onehalfspacing}


%%%%%%%%%%%%%%%%%%%%%%%%%%%%%%%%%%%%%%%%%%%%%%%%%%%
%
%\cleardoublepage
%%%% empty page
%\newpage
%\thispagestyle{plain} % empty
%\mbox{}
%%%
%
%%%%%%%%%%%%%%%%%%%%%%%%%%%%%%%%%%%%%%%%%%%%%%%%%%%


%%%%%%%%%%%%%%%%%%%%%%%%%%%%%%%%%%%%%%%%%%%%%%%%%%

\cleardoublepage
%%% empty page
\newpage
\thispagestyle{plain} % empty
\mbox{}
%%

%%%%%%%%%%%%%%%%%%%%%%%%%%%%%%%%%%%%%%%%%%%%%%%%%%

\begin{onehalfspacing}
\tableofcontents*

\cleardoublepage

\listoffigures*

\cleardoublepage

\listoftables*
\end{onehalfspacing}
%%% List of symbols and acronyms %%%

% name
%\renewcommand{\nomname}{List of symbols and acronyms}

% spacing 
\newlength{\nomitemorigsep}
\setlength{\nomitemorigsep}{\nomitemsep}
\setlength{\nomitemsep}{-\parsep}

% groups
\renewcommand{\nomgroup}[1]{
%
\itemsep\nomitemorigsep
\ifthenelse{\equal{#1}{A}}{\item[\textbf{Acronyms}]}{
%
\ifthenelse{\equal{#1}{B}}{\item[\textbf{ccc}]}{
%
\ifthenelse{\equal{#1}{K}}{\item[\textbf{vvv}]}{
}}}
\itemsep\nomitemsep}


\printnomenclature[7.6em]

\cleardoublepage
%%% empty page
%\newpage
%\thispagestyle{plain} % empty
%\mbox{}
%%


%\thispagestyle{empty}





