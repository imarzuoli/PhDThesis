\chapter{Introduction}

%PICTURES
%0 Scheme
%1 many drug delivery vehicles (collage mio)
%2 plot/table on resistant antibiotics
%3 Some structure of AmPs (beta, alpha, ...)
%4 mech of action AMPs
%5 picture of LFC e some results from experiments
%6 triskelion and capzip
%7 exp results preliminary


% REF:
% AlphaGo, AIreview, Rossi2018, history_medicine_review, other_review_early_drugs?, drug_database, review_drug_delivery, WIDES_database, ABX_database_jhopkins, Mitchess1945, Strebhardt2008.

\lettrine{``P}{hilosophical introduction"} to be finished/modified when the work is finished...

%\lettrine{T}{heory} stays to experiment as experiment stays to nature. And science stays to technology as technology stays to life.

%To navigate this huge gap is the scientist call, in an effort to bring the extremes closer, giving a model of how nature functions, or to enrich the space in between, inventing new realities. Technology bridges every day this very gap: every consequence of our abstract thinking is technology, an hidden layer of inductions and deductions which brings us from abstract principle to solutions. Computers, the ultimate technology, are emulating increasingly better and more efficiently this process, sparing us the awareness of the complex mechanisms which thread the problem to its solution. It was in the past two centuries that we witnessed such an evolution of techniques, inventions and machineries that we can now exploit years of theoretical thinking by using tools in practical problems to eventually modify nature in every daily activities. If in the past nature was the mystery and the human intervention on it was simple to understand, now on the contrary many basic principles of the physics laws are clear to most of us but human inventions became increasingly complex, condensing centuries of discoveries in simple, efficient tools. We use planes, drugs and the internet not because we perfectly understand how they work, but because we trust the collective knowledge we, as humans, have accumulated so far.

%Out of the many fields at service of the humanity wellness, the most challenging and still far away from being exhausted is the understanding and manipulation of the human body and mind. While machines - in the broadest sense possible - can perform actions in our place, they can't yet think and live on their own. It is striking how we are finally scratching the understanding of these two entities, mind and body, in the same historical moment, at a point where computers imitates the human reasoning \cite{AlphaGo, AIreview} and biological materials are turned into semi functional organs \cite{Rossi2018}. However, we are far from completing the jigsaw of knowledge on these topic. On the contrary with the progression of the techniques available to investigate various fields, we realise how vast is the space to be explored. It is then a logical consequence that the modern scientist is becoming more and more specialised, drifting away from the comprehensive knowledge owned by scientists up to two centuries ago; but exactly because the full picture is challenging, every project aimed at understanding an aspect of these enormously vast themes, no matter how tiny the subject is, is involved in a network of efforts, in the hope and trust that piecewise knowledge can build a unique and organic corpus.

%This thesis places itself in the domain of understanding how the human body works - how the (non) equilibrium of life is possible and how human intervention can be possible. The tiny and narrow topic it covers wants to explore one possible way in which we help the body to heal itself and to defend itself against external malicious agents. To correct those processes that go wrong means life, and we are biologically and emotionally pushed towards actions that prolong and improve life and have always looked at ways of curing ourselves. But the body evolved in the past blind to reason, on the contrary taking advantage of multiple defences and barrier which secured it from the failure of the reasoning, and it is know to us often a mystery, as we struggle to understand many of its components, letting aside the whole picture.

%Quite blindly then we developed in the past a medicinal science which managed to be of service to the human envelope, in what was a remarkable game of trial and error resorting to magic first and to our intelligence last \cite{history_medicine_review}. The risk and inevitable failures tracing the path were a necessary toll to the utmost necessity of keeping healthy, safe and - ultimately - alive. In a history resembling the evolution of technology, we resorted to nature for beneficial molecules \cite{above}, which we called drugs in the initial fuzziness existing between healing and deadly, but then humans started identifying the beneficial principle in these natural remedies \cite{other_review_early_drugs?}, and ultimately to produce new molecules \cite{other_review_early_drugs?}. The increasing understanding of how we worked pointed out the many challenges a drug has to withstand to be efficient. And this knowledge poses question: if so many barriers prevent a drug from entering the body, how can infectious agents have found a way to our cells? And if we want to fight those ones instead, why the body cannot recognise these helpful molecules as beneficial ones and is fighting them instead? Are they perhaps damaging for us as well, in some way we have not yet understood?

%Luckily, if knowledge has brought awareness of the complexity of the machine our body is, it started bringing also solutions. We do now have drugs \cite{drug_database}, we know how to selectively deliver some of them \cite{review_drug_delivery}. We have disinfectants \cite{WIDES_database}, and we have antibiotics to fight pathogens \cite{ABXdatabaseJhopkins}. We do know what we are composed of \cite{Mitchess1945}, and how this material rearranges in organelle, cells, organs. And we know some of the mechanisms concerting these parts together \cite{what?}.

%We are finally moving in the direction of the magic bullet envisioned a century ago by Nobel Prize Paul Ehrlich, who dreamed of a 'personalised and tailored drug' able to target specific molecular defects while being harmless - if not beneficial - to the other cells \cite{Strebhardt2008}. Such success would condense in a tiny amount of space a century of efforts in understanding the human body. But we still miss many pieces of information as the more we zoom in, the less each single researcher can monitor at once and the more we realize is there to be discovered.

%In this prospect, looking at how one particular molecule behaves with respect to a particular environment, as this thesis does, using a simplified theory (sometimes the only possible) is certainly a tiny fragment of knowledge added to the world of science. But, joint to the other scientific output from the community, it is a necessary, meaningful and promising fragment.

\vspace{1cm}
(...thus) This introduction is meant to give an overview of the many different challenges the fields of medicine and bioengineering have faced in recent years. These challenges have promoted the research on self-assembling antimicrobial peptides, despite they were not a primary source of interest in these fields, as other materials and concepts were deemed more suitable to solve the tasks coming along the way. It is therefore important to clarify the landscape of such other solutions and approaches to understand and value why a change in the research focus has come to age. Figure \ref{fig:intro} provides a work flow of this introductory chapter to help the reader in identifying the sections of interest.

\begin{figure}
\begin{center}
\Large{\textbf{Motivations of the work: a graphical abstract}}\par\bigskip
\includegraphics[width = \textwidth]{pics/scheme_intro}
\caption[Graphical abstract of introduction]{Figures a. (acetylsalicylic acid) and j. (diphenyl-alanine) in bond representation. Remaining figures adapted from: b. [--]; c. \cite{poly}; d. \cite{lipo}; e. \cite{Schoonen2014}, f. \cite{Blair2014}; g. \cite{phage}; h. \cite{Torres2019}; i. \cite{Nguyen2011}; k. \cite{Castelletto2016}} \label{fig:intro}
\end{center}
\end{figure}

\clearpage


\section{Drug delivery: challenges and solutions}
% REF
% Astruc200, Boisselier2009, Bunker2016, Buning2015, Daya2008, Depan2011, Douglas2012, Erol2017, Fan2017, Habibi2016, HarshaRao2018, Hughes2005, Jain2016, Jain2017, Jiang2012, Julien2013, Kawakatsu2004, King2014, Krol2012, Lakshmi2007, Lammers2009, Lauer2017, Liechty2010, Linko2015, Lobo2009, Ma2012, Malay2019, Masaoka2006, Mitragotri2014, Nair2007, Nicolas2013, Pattni2015, Pattni2015paper, Rao2018, Silva2013, Singh2018, Smalley2017, Wu2009, Yan2010, Yeates2019, Yingchoncharoen2016, Yoo2011, Zhang2014

\subsection{Environmental challenges of drug delivery}
The problem of drug delivery is an excellent example of the hurdles existing between a theoretical reasoning and nature: a new drug is usually designed to affect a specific target. Even if in silico experiment can prove the efficacy, its usefulness is bound to the ability to cross the many barriers dividing the inoculation site from the very target inside the human body.
%
To reach the aimed organ, drug molecules must be compatible with the different cellular environments they cross but be preferentially retained, and act only on the ones they are designed for. This implies a subtle balance between a disruptive activity on one side, and harmlessness on the other, least the compound is recognised as dangerous and disposed of by the efficient immune and reticuloendothelial systems of the body which aim at neutralise exogenous substances.

As en example, the trip of an orally  administered ``free" drug, i.e.\ an active molecule without any aiding delivery agent, passes through the digestive system, with its challenging acidic environment and limited permeation across the intestinal epithelium, and from there to the blood stream \cite{Masaoka2006, Mitragotri2014}. The drug then diffuses in the tissues flanking the blood vessels naturally depleting its concentration downstream \cite{Krol2012}, so that regions further away in the line have less chances of getting a sizeable dose. This implies that high drug concentrations might be needed at the starting point to efficiently target every organ.

However, this naive picture of a drug diffusing in the body is complicated by the impermeability of specific tissues: the brain for example, one of the most delicate organs in the body, is well protected from the attack of external agents by the blood brain barrier (BBB), which allows the passage of small molecules only ($<$ 400-500 Da, while standard 'small molecule drugs' weigh up to 900 Da) which, additionally, have high lipid solubility \cite{Pattni2015, Krol2012}. Other tissues, like tumoral ones, are instead poorly vasculated, reducing the chances of delivery at their interior \cite{Pattni2015}.

Moreover, as already hinted, during their journey to reach the receptor, enzyme or organelle it is meant for, a drug must not be sequestered by the immune systems. However, many inorganic small molecules are not mimetic by themselves, i.e.\ often they do not resemble the ones naturally present in the body, and this brings uncertainty on how they would interact with them. Generally, as soon as they reach the blood stream, small molecules are coated by a protein corona based on their shape and charge \cite{Krol2012}. Such modifications are difficult to predict and can disrupt or decrease significantly the efficacy of a compound as they modify the way drugs are recognised and absorbed by the target. 

For all the above reasons, research has focussed on developing systems to assist the delivery of drugs \cite{Jain2016, Pattni2015, Mitragotri2014}. A mimetic carrier can not only improve delivery, but also be designed to selectively bind to particular tissues or to trigger the drug release after a given time or upon changes in environmental variables (for example pH) to reduce drug concentration in non targeted regions. A stand alone field of research has then focussed on the development of delivery vehicles irrespective from the quest for disease targeting compounds. The optimised products of the two efforts can then be paired according to the condition to address.

At present, many molecules have been successfully employed to build drug vehicles: inorganic metals, polymers, lipids and proteins are all suitable for the aim and offer a range of different physico-chemical characteristics useful to target different cells \cite{Hughes2005} (Figure \ref{fig:vehicles}). A brief (and non exhaustive) overview of some of them is meaningful to point out the broad variety and exoticity of structures which can be useful, sometimes unexpectedly, to the medical world.


\subsection{Inorganic materials for small drugs delivery}

\paragraph{Metal nanoparticles} In the range of inorganic compounds, golden nanoparticles demonstrated to be remarkable for tumour treatment: first of all, they can be customised in shape and size (down to a 10 nm radius), made less visible to the immune system by coating with biologically active moieties or by conjugation to a poly-ethyleneglycol (PEG) polymer layer \cite{Singh2018}. Moreover, they possess optical properties that allow to track them inside the body, they can be thermally stimulated to trigger drug release, favour the penetration through cell membrane or disrupt the cells nearby \cite{Boisselier2009}. At present, there are mixed evidence about their toxicity \cite{Boisselier2009} and doubts have been raised on the long term effects of metallic fragments in the body. For that reason, only a few golden nanoparticle based compounds have made to the clinical stage so far \cite{Singh2018} but, given their high and still unexplored potential, they continue to be a primary interest for the medical community and a very active research field.

\begin{figure}
\begin{center}
\includegraphics[width = 0.5\textwidth]{pics/carbon_review.jpg}
\vspace{0.2cm}
\caption[Materials for drug delivery vehicles]{INCLUDE ONE EXAMPLE IMAGE (FROM PAPERS) FOR EACH MATERIAL (nanopartcles, carbon, polymers, lipids, DNA, proteins).} \label{fig:vehicles}
\end{center}
\end{figure}

\paragraph{Carbon nanotubes}
Similarly, carbon nanotubes have been used for biomedical applications as they have a high loading efficiency thanks to their high surface area and easy interaction with biomolecules through van der Waals interactions, $\pi$-$\pi$ stacking or hydrophobic effect \cite{Erol2017}. Therefore they are easy to functionalise through conjugation to extra organic groups to increase their biocompatibility and have potential for targeted drug release upon change in environmental pH \cite{Depan2011}.

\paragraph{Polymers} Polymers are another large class of inorganic molecules functionalised for the benefit of medicine: PEG for example has already been mentioned as aid to make golden nanoparticles bio-compatible. Indeed, thanks to its high hydrophilicity, it is a clinically approved molecule widely used to mimetise structures (e.g.\ inorganic nanoparticles, peptides) which in turn carry a drug \cite{Lammers2009}; or as a stand alone carrier system which has a high drug payload \cite{Liechty2010}. The great strength of polymers is their flexibility: as each of their constituent monomer can be either hydrophilic or hydrophobic, they can be engineered to assemble in many different structures\cite{Kawakatsu2004}. Moreover, they can trigger a sustained drug release by swelling slowly in water,\cite{Nicolas2013} or undergo sol-gel phase transition upon specific changes in the environment \cite{Liechty2010}. Finally, research has also focussed on improving their biodegradability \cite{Nair2007} and in making polymers a bioactive compound itself \cite{Rao2018}.


\subsection{Organic materials for small drugs delivery} \label{sec:organic}

A somehow opposite approach for designing drug vehicles consists in using molecules similar to the ones present in the body, in an effort to exploit already available biocompatible materials and reduce toxicity \cite{Yoo2011}. In this category fall lipids, DNA and peptides.

\paragraph{Lipids} Lipids are the main constituents of the cell membrane. They come with a great variety, enhanced by the many species produced synthetically. The components selected for drug delivery are usually taken from the biological lipidome \cite{seeRebuttal}, but their composition differs from the one of cellular membranes, and possibly includes synthetic molecules, to tune the release properties and enable them to survive the delivery journey \cite{Yingchoncharoen2016}. Lipids can encapsulate efficiently both hydrophobic or hydrophilic drugs, arranging themselves respectively in micelles structures (monolayer spheres with the hydrophobic tails facing the interior) or in liposomes (bilayer spheres with a water filled core) \cite{Bunker2016}. By now, many of them had overcome the clinical stage and are currently approved for cancer and infections \cite{Pattni2015paper, Jain2017}.

\paragraph{DNA scaffolds} Similarly, many DNA scaffolds have been tested for smart delivery: DNA origami is nowadays an established technique to build three dimensional customised solids \cite{Linko2015}, and the nanometric knowledge about their constituents makes possible fine tuning them for a triggered release of the content \cite{Douglas2012}. First studies proved them successful in delivering anticancer agents \cite{Zhang2014, Jiang2012}, however they are very sensitive to cellular environment and this, united with high production costs and the relative young developments in their manipulation, prevented them to constitute a viable class of carriers so far.

\paragraph{Peptidic scaffolds} Another widely used and trustworthy mimetic vehicle comes, quite surprisingly, from the world of pathogens: viruses have co-evolved with humans, to be able to penetrate into cells where they complete their reproductive cycle \cite{Lobo2009}. Therefore their capsid, the peptidic shell encapsulating the genome, is highly suitable for cell penetration. The first application sought historically was to employ genome free viruses to stimulate and train the natural immune response against the respective genome-loaded ones, creating viral vaccines - in a similar fashion to what already done with the inoculation of dead bacteria to counteract the infections caused from them \cite{Lauer2017}.
Later in the history, their potential as cargo carrier was pursued first by modifying the original genetic material to include sequences beneficial for the host cell, and to prevent the infectious duplication at the same time. In particular adeno-associated virus (AAV) has been widely studied \cite{Daya2008} as it triggers a low immune response \cite{Buning2015}, and the first AAV viral therapy has been finally approved a few years ago \cite{Smalley2017}.
%
To fully exploit the potential of a peptidic carrier many efforts have focussed on synthesising in vitro gene-free capsids, either as they appear in nature \cite{Wu2009} or designing artificial building blocks, which assemble in so called Virus-Like particles (VLPs), to help overcoming the reaction stimulated by specific viral capsids to which the immune system is sensible to.
%
Similarly to other delivery vehicles, the surface of VLPs can be functionalised with additional molecules to improve the target selectivity and increase biocompatibility, while the peptidic scaffold grants robustness to the structure. Therefore, VLPs loaded with drugs can be tuned for an efficient intra cellular release \cite{Ma2012}.

Among their advantages, peptides present biocompatibility, a low production cost and a tunable bioactivity thanks to their chemical diversity, which helps in tailor the assembly toward the target of interest \cite{Fan2017}. Moreover, the variety of amino acid available makes possible to load peptidic structures with both hydrophilic and hydrophobic drugs, according to their amino acid composition \cite{Habibi2016,Ma2012}.
%
The easy manipulation of peptidic structure derives from the fact that proteins are a fundamental component of the human body, so that there is a vast literature on their interactions with membranes, cell receptors and in general biological components, from which the design for novel materials can take inspiration to employ building blocks sensible to particular triggers within the body.

A step further in engineering peptidic structures is represented by the design of self-assembling functional structures from first principles, exploiting the physico chemical characteristics of peptides, regardless their resemblance of viral capsids.
%
Indeed, self-assembling peptides can form nanostructures ranging from nanoparticles to nanotubes, nanofibers, nanorods and hydrogels \cite{Fan2017,Habibi2016}. The assembly is modulated by the peptide length and its hydrophobic or hydrophilic character: on one end of the length scale, phenylalanine dipeptides were designed with inspiration from a pathogenic pathway of molecular self-assembly \cite{Yan2010} and were shown to self-assemble in a multiscale process producing nanotubes able to load drug molecules \cite{Silva2013}. The relatively small diphenylalanine building block is non the less complex as it bears two charged termini (as the process is observed at neutral pH), and two aromatic hydrophobic rings, so that the dipeptide is driven towards assembly by the hydrophobic forces acting on the phenylalanine side chains and the complementary charges of the termini.

In a different approach, longer sequences can be employed to guide the formation of the local structure, as they organise spatially in well studied motives (the secondary structure) with known interactions among themselves.
%
The two typical secondary structures, $\alpha$-helices and $\beta$-sheets, appear in sequences of about 20 or more amino acids length and can both be amphiphatic, thus promoting the assembly between the hydrophobic faces of different copies of the same structure. With the appearance of a secondary structure, more complex building blocks can be designed, to tune the shape into the ones needed for the supramolecular organisation of interest \cite{King2014}. Again, the knowledge of many protein structures\cite{PDB} give us insight in how the small ones can hierarchically assemble - however the challenge and outlook often goes in the direction of synthesising exotic, non natural, novel geometries \cite{Yeates2019,Malay2019}.


\section{Antimicrobial resistance}

% REF
% Abouzeed2008, Abraham1988, Baroud2013, Baucheron2014, Billal2011, Birkegard2018, Blair2014, Delcour2009, Dolejska2013, Floyd2010, Gao2010, Gordon2000, Hirakawa2004, Hu2012, Kapoor2017, Katayama2000, Kim2013, Kojima2013, Kumar2014, Lavigne2013, Livermore2008, Long2006, Lynch2013, McKellar1999, Niewiadomska2019, Nikaido2011, Nordmann2011, Norris2013, Novais2012, Ogawa2012, Oneill2016, Poulou2013, Qin2012, Queenan2007, Queenan2010, Santos2017, Shore2011, Tamber2003, Tangden2013, Tran2013, Tzouvelekis2012, Vargiu2012, Vetting2011, Voulgari2013, Woodford2011, Woodford2013, Wozniak2012, Wright2005

The previous brief review on drug carriers rotates around the paradigm that a drug is a small inorganic compound (of mass up to 900 Da) which targets a specific molecule of a specific target of a mammal or bacterial cell. In this light, the ultimate goal of the delivery vehicle is to carry the drug to the site of action where it can interfere with the processes it is assigned to. Very often the target of interest of small molecule drugs are proteins: out of the 695 small drugs approved by FDA (the American Food and Drug Administration agency) to target human molecules, 667 acts on proteins. Similarly, 189 of the 198 small drugs approved to treat pathogens have a protein as their target \cite{Santos2017}.
%
(It must be noticed however that the identification of an unambiguous drug target poses challenges in many cases, especially when the drug binds to a protein complex or to a number of closely related gene products \cite{Santos2017}.)

In presenting the aforementioned figures, the data were naturally split among the drugs which target human molecules, ``repairing" some faulty process in the human body, or the ones active against bacteria, which ``disrupts" the bacterium life cycle in order to kill or prevent the reproduction of the pathogen.
%
It appears evident that the pool of drugs available to the second purpose are in consistently lower number than the ones addressing human molecules. This comes from the nature of the action they perform: molecules targeting human proteins need to be highly specific to avoid interference with other proteins or with healthy cells, and in a sufficient number to address the variety of diseases affecting the human body.
%
Antibiotic must be non-toxic for human cells as well, i.e.\ their target must not be shared between mammal and bacterial cells \cite{???}, but there is a less stringent requirement in their selectivity on different bacterial species. On the contrary, it is often useful to have a broad-spectrum compound. This cross-species efficacy and non-toxic property is obtained thanks to the evolutionary relationship among bacterial species, and between bacteria and humans: while the first are closely related, and therefore share homologous proteins with very similar structures, humans have less architectures in common with them, allowing for a resilience against bacteria-targeting drugs \cite{???}.
%
Of course the set of bacterial species is very diverse and the cross-species effectiveness of some drugs does not extend to the whole bacterial population. This demonstrates to be a positive feature, given the large amount of beneficial bacteria that live in symbiosis with the human body (especially in the gut \cite{???}) and that must be preserved for an optimal wellness.

In the framework described above, it is understandable that first-time research on antibiotics was satisfied with the development of a handful of potent, broad-spectrum compounds.
%
Penicillin, the first of them, was isolated from a mould in 1928 by Alexander Fleming. It acts inhibiting the formation of peptidoglycan cross-links in the bacterial cell wall (for further details on bacterial cell membrane structure the reader can refer to Section \ref{sec:host-defense-peptides} and the relative references). This inhibition is achieved through binding to the enzyme DD-transpeptidase responsible for the catalysis of such cross-link \cite{Gordon2000}.
%
As foreseen from Fleming himself in his Nobel Prize acceptance speech, bacteria can become immune to penicillin, and this is specifically achieved by either production of penicillase, an enzyme that degrades penicillin, or by subtle changes in the structure of the penicillin-binding proteins to prevent penicillin binding, or again by removal of the drug outside of the cell through specially re-purposed efflux pumps that they use to release substances from the cell \cite{???}.

\subsection{Course of antimicrobial resistance} \label{sec:course_AMR}
This mechanisms are not an exceptional characteristic of penicillin, and many drugs lost their effectiveness against some bacteria since their discovery till nowadays. Indeed the antibiotic landscape is a dynamic entity in which newly discovered ones enter while others exit after having been exploited for years.

In the first stages of the insurgence of antimicrobial resistance (AMR), less effectiveness of a drug means that some strains of bacteria are not damaged by the standard doses of the drug as they possess some natural occurring mutations in their genome promoting an escape mechanism which invalidate the drug effectiveness \cite{Kapoor2017,Blair2014}. Usually only a small population of bacteria is resistant, and it can be killed never the less by augmenting the dose of the drug. However, the natural course of AMR states that the resistant population will replicate faster that the peers of the same species because it is more fit in an environment challenged by the presence of the drug. It is noteworthy that this fitness might not be optimal in a natural drug-free environment - and indeed the wild population has not been selected for that genotype - but under the pressure derived from the treatment, other characteristics result more advantageous.
%
In the short time scale it is usually sufficient to increase the doses of a drug to re-gain efficiency against the target, but it has been observed that a species resistant to a drug can usually adapt to higher doses of the same \cite{????}. Moreover, high drug doses are not always applicable due to the severe side effects they are connected to \cite{????}.

The spread of resistance between bacterial cells and even species is very effective as bacteria are able to exchange genetic material with other individuals via small rings of DNA in a process called conjugation \cite{????}. In this way the advantageous characters spread across individuals, and species with an innate resistance can transfer to other ones their mechanisms of resilience to a particular drug.
%
Therefore, despite AMR is an evolutionary mechanism, the fast pace at which bacteria replicates, their enormous population (in terms of individuals), and the relative easy horizontal gene transfer through conjugation place the insurgence of resistance well within the human lifespan time scale \cite{????}.

It is then clear this very complex problem depends on many variables: the casual appearance of resistant individuals, the transfer of information between them, the relatively larger fitness of resistant individuals and the dosage and time line of the drug administration. Many mathematical models have been implemented to understand the issue \cite{Birkegard2018,Niewiadomska2019}, but it is known that some particular strategies of drug administration are worse than other, favouring the proliferation of so called ``super" bugs.
%
One example of bad administration strategy is the underdosage of antibiotics: a low drug load is likely to harm but not kill pathogens, in particular to promote the fitness of resistant ones. In a sort of ``gym" or ``vaccination" process for bacteria, an underdosed drug would kill the weakest individuals but strengthen the resistant population, which would now be fitted to the challenges of higher doses \cite{????}.
%
Similarly, the abuse of antibiotics puts an high pressure on the pathogenic populations, which is desirable but at the same time can induce a faster emergence of escape mechanisms \cite{????}.
%
In this context it must be noticed that many drugs are bacteriostatic agent as opposed to bactericidal: i.e.\ they prevent the bacterium growth rather than kill it, as they are meant to control the bacteria reproduction and slow down the damage while host defence mechanisms eradicate them.
%
Thus if an high dosage of a bactericidal agent may extinguish the bacterial population and eradicate the disease, for bacteriostatic drugs, once they are removed, bacteria can usually start again the reproduction cycle.

Finally, it is noteworthy that abuse of antibiotics can take many forms: a part from the drugs used to treat human related diseases, the agricultural and breeding sectors are constantly using antibiotics to keep their products secure form illness. This results in large quantities of drugs to be released in the soil and water, which ultimately reach humans in underdosed quantities. Diseases of plants and animals are different from the ones affecting humans, however some drugs are effective on many bacteria including the one harming humans. Therefore the widespread use of antibiotic for animals or plants can ultimately train resistant bacteria in humans \cite{????}.
%
Additionally, diseases can cross species: this means that an extra care must be taken in the treatment of non human bacteria least to promote resistant ones which can at a point cross species and affect us \cite{????}.

The complexity and severity of the issue is such that it has been raised to the status of national emergency in several countries, including UK, as we are leaving the century in which antibiotics were discovered, to enter a phase in which we count the number of the ones loosing efficacy \cite{Oneill2016}.


\subsection{Mechanisms of antimicrobial resistance to small drugs} \label{sec:AMR_mechs}

Antimicrobial resistance can manifest through many different mechanisms, as highlighted in the example of the penicillin resistant bacteria.
%
In particular, resistance mechanisms fall into three main groups: a first group minimises intracellular concentration of the antibiotic preventing penetration or maximising efflux; a second one modifies the antibiotic target by genetic mutation or post-translational modification; finally a third group inactivates the antibiotic by hydrolysis or modification of the drug molecule (Figure \ref{fig:amr}) \cite{Blair2014}.
\begin{figure}[h]
\begin{center}
\includegraphics[width = 0.6\textwidth]{pics/amr1}
%\includegraphics[width = 0.4\textwidth]{pics/amr2}
\vspace{0.8cm}
\includegraphics[height = 0.36\textheight]{pics/amr3}
\hspace{0.5cm}
\includegraphics[height = 0.36\textheight]{pics/amr4_half}
\caption[Mechanisms of antimicrobial resistance to small drugs]{Mechanisms of antimicrobial resistance to small drugs. a) Intrinsic mechanisms of resistance (removal of antibiotic B by efflux pump and inaccessibility of antibiotic C to the PBP target because of membrane impermeability). b) Target site change via mutation or protection. c) Direct interactions with antibiotics causing its disruption or structural modification. Reproduced from \cite{Blair2014}.} \label{fig:amr}
\end{center}
\end{figure}


\paragraph{Prevention of access to target}
One possible mechanism of defence bacteria employ against antibiotics is to prevent the access to the target. This is performed either blocking the drug influx or promoting its quick efflux in the eventuality it has entered the cell.

Regarding drug influx, not all the molecules can enter the cell permeating the membrane, and this holds particularly for hydrophilic antibiotics tackling Gram-negative bacteria: indeed, compared with Gram-positive ones, Gram-negative bacteria are intrinsically less permeable because of the structure of the additional outer membrane \cite{Delcour2009}, therefore hydrophilic molecules are imported into the cell through outer-membrane porin proteins \cite{Vargiu2012,Kojima2013} (for further details on the bacterial membrane structure, see Section \ref{sec:host-defense-peptides}).
%
The major porins of most Enterobacteriaceae are thought to be non-specific channels \cite{Tran2013}, therefore replacing porins with more selective channels or down regulating their expression would limit the intake of the drug. This last mechanism is well established and contribute to resistance to many different drugs in Gram-negative bacteria, including newer drugs such as carbapenems and cephalosporins, for which resistance is otherwise mediated by enzymatic degradation \cite{Tamber2003,Baroud2013,Lavigne2013,Poulou2013,Wozniak2012} (see the related paragraph below).
Alternatively, as happens in E. coli exposed to carbapenems for example, not only the porin expression is down-regulated, but also the genes coding for porins are heavily mutated, suggesting that changes in the porin structure can enhance their selectivity and reduce the drug influx \cite{Lavigne2013,Novais2012,Tangden2013}.

A strategy complementary to prevent drug influx is to dispose the drug efficiently once it has invaded the cell. Bacterial efflux pumps transport many antibiotics out of the cell, and they constitute a major hurdle for the treatment of Gram-negative bacteria as opposed to Gram-positive ones. Indeed, many of the drugs effective for the latters are evacuated by the formers through efflux pumps; in particular, multidrug resistance (MDR) efflux pumps can transport a wide range of structurally dissimilar substrates.
%
All bacteria can produce their own MDR pumps \cite{Floyd2010,Hu2012,Kim2013,Ogawa2012}, but it has also been shown that the some of the genes encoding for them have been transferred to plasmids and thus can be transferred to other bacterial species, disseminating resistance \cite{Dolejska2013}.
%
The over-expression of efflux pump seen in multidrug-resistant bacteria is often due to mutation in the regulatory network controlling it, \cite{Abouzeed2008}, but it can also occur as a result of induction in response to environmental signals and in conditions in which their function is required \cite{Baucheron2014,Nikaido2011,Hirakawa2004}.


\paragraph{Change or modification of the antibiotic target}
Most antibiotics bind to the target with high affinity and therefore specificity. Small modifications of the target structure can disrupt an efficient binding of the antibiotic, still allowing the target to maintain its normal function. Preservation of the target from the action of a drug can be reached by either mutation or protection of the binding site.

In the first case, a casual mutation would provide such minimal required change and the resistant population would spread according to its improved fitness.
%
An example is the development of resistance to linezolid in S. pneumoniae and S. aureus: this drugs targets the 23S rRNA ribosomal subunit of Gram-positive bacteria which is encoded by multiple, identical copies of its gene. The use of linezolid has selected first a population with a mutation in one of the copies, which has afterwards passed to other copies via recombination, generating a population favouring the synthesis of the mutant subunit \cite{Billal2011,Gao2010}.
%
Other examples include mutations occurred by transformation, i.e.\ uptake of DNA from the environment: in the case of penicillin resistant S. pneumoniae, a penicillin-binding protein gene is included in the genome by recombination with DNA from the closely related species Streptococcus mitis. Similarly, the acquisition of a gene homologous resulted in a methicillin-resistant strain of S. aureus \cite{Shore2011}: this gene allows the synthesis of the PBP2 (penicillin binding protein 2) protein which enable cell wall synthesis despite the native PBP is inhibited by the antibiotic \cite{Katayama2000}.

The second mechanism modification mentioned consists in protecting the target from the binding of the drug via addition of chemical groups to the target after its synthesis; as such, these modifications do not require mutations at the genetic level.
%
Methylation is an important process which triggers resistance: for example, under the pressure of macrolides, lincosamines and streptogramins, the 16S rRNA subunit is methylated and thus the drug-binding site altered \cite{Kumar2014}. Similarly, it has found that specific methylation of a base (A2503) in the 23S rRNA subunit confers resistance to many drugs that target nearby regions (phenicols, pleuromutilins, streptogramins, lincosamides and oxazolidonones) \cite{Long2006}.
%
In a different mechanism, quinolone resistance can be conferred by a gene coding for a pentapeptide repeat proteins (PRPs), which binds to topoisomerase IV and DNA gyrase promoting the release of the drug and rescuing the normal function of topoisomerase \cite{Vetting2011}.

\paragraph{Direct modification of antibiotics}
Finally, bacteria can modify or destroy drugs to prevent their action, usually by either hydrolysis or by transfer of a chemical group.
%
Enzyme-catalysed modification of antibiotics is a major mechanism of antibiotic resistance: the very first example being penicillinase (a $\beta$-lactamase) which destroy penicillin \cite{Abraham1988}.
%
Since this discovery, thousands of enzymes have been identified that can degrade and modify antibiotics of different classes, such as $\beta$-lactams, aminoglycosides, phenicols and macrolides \cite{Livermore2008,Nordmann2011,Voulgari2013,Woodford2011}.
%
These enzymes co-evolved together with the newly developed drugs which bacteria are exposed to, to include in their spectrum of disruptive action new compounds of similar composition: for example the first $\beta$-lactmases evolved to be active against the new $\beta$-lactams antibiotics developed, up to the emergence of isolates resistant to all the drugs in the $\beta$-lactam class \cite{Woodford2013}.
%
This localised emergence of resistance is a particularly serious problem because of the effectiveness with which these mechanisms spread to the whole bacterial population in a short period of time \cite{Voulgari2013,Woodford2013,Lynch2013}.
%
%Moreover, as hinted in Section \ref{sec:course_AMR}, the inefficacy of one class of drugs brings inevitably to a more massive use of other compounds (for example carbapenem in replacement of $\beta$-lactams), to which bacteria develop resistance (in the example above, developing the so called carbapenemases to hydrolyse the drug) \cite{Queenan2007,Queenan2010,Tzouvelekis2012}.

The addition by bacterial enzymes of chemical groups (for example acyl, phosphate, nucleotidyl or ribitoyl \cite{Wright2005}) to vulnerable sites on the antibiotic molecule is another mechanism to block the action of the drug, as it prevents its binding to the target protein due to steric hindrance.
%
Antibiotics constituted by large molecules with many exposed hydroxyl and amide groups are particularly susceptible to these modifications. An example of such antibiotics is the aminoglycoside class (in which streptomycin is included), which can be modified by three classes of enzymes, grouped according to the chemical moiety added: acetyltransferases, phosphotransferases and nucleotidyltransferases \cite{Norris2013}.
%
A recent development reports the discovery of a genetic island in Campylobacter coli isolated from broiler chickens in China coding for six of these enzymes, including members of all three classes: the expression of such genes would then confer resistance to many antibiotic of the aminoglycoside class at once \cite{Qin2012}.

\hspace{0.5cm}
\\
All together, the recent progress in understanding the mechanisms of antimicrobial resistance has helped in directing the development of new drugs, in particular the modification and the improvement of existing compounds to escape the resistance developed by bacteria. This in turn has highlight the effectiveness of some clinical strategies, such as the use of combined therapies, to counteract an early development of resistance. However, the problem persists and more knowledge needs to be gather for a complete understanding and the possible development of resistance-free compounds.


\section{Alternative antibiotic strategies: antimicrobial peptides}

% REFS
% APD3,DBAASP2,dbAMP,antiBP2,amPEP
% Allen1998, Avrahami2004, Bahar2013, Band2014, Baranska-Rybak2006, Batoni2011, Bertelsen2012, Biegeleisen2006, Bishop2000, Bolintineanu2011, Bokarewa2004, Brogden2005, Brotz1998, Bugg1991, Campos2004, Chan2006, Cherkasov2009, Choudhary2011, Cipcigan2018, Costerton1999, Dathe1999, Deslouches2005, DiLuca2014, Ebbensgaard2015, Ebenhan2014, Fabretti2006, Farrotti2017, Fjell2011, Frick2011, Fujii1993, Gazit1995, Gifford2005, Glukhov2005, Gonzalez2008, Grieco2011, Groisman1992, Gunn1998, Guo1998, Hallock2003, Hancock2006, He1995, Hilpert2005, Hilpert2006, Huang2010, Hwang1998, Jiang2011, Jin2004, Jolivet-Gougeon2014, Joo2012, Joo2016, Juhas2015, Kagan1990, Khandelia2005, Klein2009, Kumariya2015, Ladokhin2001, Lee2004, Lee2017, Lehrer2004, Lewenza2005, Lin2016, Liu2018, Llobet2008, Loose2006, Loutet2006, Mah2001, Mahlapuu2016, Malmsten2016, Mantravadi2019, Mason2005, Matsuzaki1996, Matsuzaki1997, Matsuzaki1998, Matsuzaki2009, Migon2018, Moskowitz2004, Naafs2018, Nelson2011, Nguyen2005, Nguyen2011, Nickel1985, Oren1998, Pag2008, Parra-Lopez1993, Peschel1999, Peschel2001, Peschel2006, Radzishevsky2005, Raetz2007, Saar-Dover2012, Schmidt2010, Schmidtchen2002, Schmitt2016, Shai1999, Schibli1999, Selsted1989, Serrano2009, Sieprawska-Lupa2004, Silhavy2010, Sol2014, Spaar2004, Spector1985, Strempel2014, Stumpe1998, Takeuchi2004, Tang2009, Taggart2003, Teufel1993, Toke2005, Tomita1994, Tsai2009, vanMeer2008, Vuong2004PIA, Vuong2004, Walters2005, Wang2004, Wang2004MsbA, Wang2006, Wang2015, Wimley2010, Wipf2009, Wiradharma2011, Yamaguchi2001, Xie2018, Yang2001, Yeaman2003, Yonezawa1992, Zasloff2002, Zhang2001, Zhang2016

In the landscape sketched above it is evident that the development of novel drugs is of crucial importance. Even more beneficial would be to have at disposal a new paradigm for their design, in order to attack pathogens in a completely novel way, avoiding to target pathways which are recognised to easily lead to the development of antimicrobial resistance.

Possible solutions are the use antibodies, bacteriophages or antimicrobial peptides instead of small molecules \cite{Mantravadi2019}.
%
Regarding antibodies, the development of pathogen-specific monoclonal antibodies (mAb) is an emerging area of research. They can be employed for example for immunisation though serum therapy, i.e. exposing the patient to the serum of an individual already immunised. Such passive immunization has been used for the treatment of bacterial infections well before the discovery and development of antibiotics, but has since then been overshadowed by the use of small-molecule compounds and is now gaining new relevance.
%
The second class mentioned, bacteriophages, are viruses which infect bacteria and archea rather than eukarya. They are effective as they can be used both in natural environmental reservoirs and in humans and are usually highly specific for one bacterial strain.
%
Both these strategies have been only partially explored so far, bringing potential for new therapies. Phage therapy is also promising in terms of promoting a low resistance development: indeed phages and bacteria have been coexisting since a long time - in evolutionary scale - and the formers are never the less effective against the latter, suggesting that their mechanism of attack is weakly prone to provoke the insurgence of resistance.

But are antimicrobial peptides the focus of this thesis: we have already highlighted the importance of peptides as tunable structural elements of drug delivery vehicles. However, they can have a role against bacteria as drug themselves when their sequence possesses some specific characteristics: such sequences, capable of damaging and/or killing bacteria, are referred to as antimicrobial peptides. The following paragraphs will explore their characteristics, modes of action and the response of bacteria against them: indeed it is crucial to understand the complexity of the picture of what is already known versus the questions that are still open. This holds in particular when the investigation proceeds by the use of simplified models, as meaningful results can proceed only if such modelling is performed in a sensible and informed fashion.

\begin{figure}
\begin{center}
\includegraphics[width = 0.7\textwidth]{pics/amp_mech.jpg}
\caption[Antimicrobial peptides]{Events occurring at the bacterial cytoplasmic membrane following initial antimicrobial peptide (AMP) adsorption. Reproduced from \cite{Nguyen2011}.} \label{fig:amp}
\end{center}
\end{figure}


\subsection{Host-defence, membrane active peptides} \label{sec:host-defense-peptides}
Antimicrobial peptides (AMPs) are naturally produced by the human body (more in general by mammal), either as stand-alone sequences or embedded in larger proteins, as first, weak and broad-spectrum defence against bacteria \cite{Nguyen2011,Bahar2013,Mahlapuu2016,Zhang2016}.
%
Thus this pool of molecules has been selected though evolution to be active against pathogens, suggesting that they are weakly prone to provoke resistance reactions in the microbes they attack.

To exploit their potential and engineer AMP-like molecules, a careful characterisation and classification of such peptides must be done. This task has been carried on in the past decades but it is not trivial, so that up to date there are many peptides with ascertained antimicrobial activity for which the mode of action is still not fully understood \cite{Ebbensgaard2015}. However, some general characteristics of these sequences and the mechanisms they employ have emerged.
%
Unsurprisingly, AMPs are heterogeneous in shape, targets and mode of action. The size can vary between 6 and 59 amino acids \cite{Brogden2005}: despite being small with respect to the average size of a protein in the human body, these macromolecules are hundreds of times larger than small molecule drugs and as such they penetrate and act on bacteria differently.

The most common target of AMPs is the bacterial membrane. Many of them cause disruption of the physical integrity of the microbial membrane while others translocate into the cytoplasm to act on intracellular targets, and the combination of the two is not uncommon either \cite{Hancock2006} (Figure \ref{fig:amp}); in general, it is widely accepted that membrane interaction is a key factor for the direct antimicrobial activity of AMPs \cite{Nguyen2011,Malmsten2016}.
%
The determinant driving the interaction between the two is the positive charge that many AMPs presents, opposed to the negative charge of bacterial membrane \cite{Zhang2001,Schmitt2016}.
%
It is striking that such simple mechanism, based on the presence of a certain number of negatively charged lipids, holds across many bacterial species despite the great variability found in their membrane composition.
%
Indeed, based on the differences in their cell envelope structure, bacteria are classified into two macro families, Gram-positive and Gram-negative. In Gram-positive bacteria, the cytoplasmic membrane is surrounded by a thick peptidoglycan layer, while for Gram-negative bacteria this membrane (which assumes the name of internal one) is surrounded by a thin peptidoglycan layer as well as an outer membrane \cite{Lin2016}.
%
The cytoplasmic membranes of both Gram-positive and Gram-negative bacteria are rich in phospholipids like phosphatidylglycerol, cardiolipin, and phosphatidylserine, which have negatively charged headgroups, highly attractive for positively charged AMPs, and this is often sufficient to promote the preferential interaction between this membrane and the peptides.

The fact that AMPs tackle negatively charged membranes is crucial for their selectivity, i.e. the fact that they do not disrupt the (mammal) cells they are produced from \cite{Glukhov2005}. Indeed, mammalian cells have a different membrane composition, in particular the mammalian membrane is rich in zwitterionic phospholipids such as phosphatidylethanolamine, phosphatidylcholine, and sphingomyelin, providing a neutral net charge \cite{Spector1985,vanMeer2008}.
%
Strictly speaking, some negatively charged lipids are present in some mammal cell types, however they are located in the inner leaflet, while the zwitterionic phospholipids are more abundant in the outer leaflet, giving an asymmetric composition \cite{???}.
%
This structures promotes weaker interaction between AMPs and the mammalian cell membrane with respect to the bacterial one as the former is driven mainly by hydrophobic interactions, while the latter by electrostatic ones.
%
Furthermore, the mammalian cell membrane has a high cholesterol content \cite{Yeaman2003, Lai2009}, which is proposed to stabilise the membrane enhancing its fluidity, so that it is more able to accommodate the perturbations caused by AMPs \cite{Zasloff2002}.

Finally, bacterial cells have a typical transmembrane potential - the difference of electrostatic potential between the inside and the outside environment - between $-130$ and $-150$ mV, while mammalian cells between $-90$ and $-110$ mV \cite{Yeaman2003,Matsuzaki2009,Ebenhan2014}.
%
Given that a potential generates an electric field across the membrane, the higher the potential, the higher the electric field pointing from outside to inside the cell. A field in such direction pushes cationic compounds on the outside of the membrane toward the membrane itself. Therefore the stronger bacterial transmembrane potential may promote an enhanced - and thus disruptive - interaction with the cells, contributing to the selectivity of AMPs between bacteria versus mammals \cite{Yeaman2003}.

\subsection{Common mechanisms of action of AMPs} \label{AMP_mechs}
Investigating the perturbation and disruption of a bacterial membrane by antimicrobial peptides is a key point of this work, therefore it is important to highlight the mechanisms known so far through which AMPs reach this outcome.
%
As already mentioned, many AMPs have a positive charge which facilitates the binding to the membrane via charge-charge recognition; accordingly, Arginine and Lysine residues are usually abundant in AMPs sequences. However, the disruptive action takes place through the interaction of the AMP with the hydrophobic core of the membrane, therefore AMPs contain also hydrophobic aromatic residues, especially Tryptophan, which favours the anchoring to the lipid core \cite{Chan2006}.
%
Overall, AMPs resort often to adopt an amphiphatic structure to segregate the hydrophilic from the hydrophobic amino acids and thus act at the interface between membrane and solution. It is interesting to notice that some of them fold into the active structure only nearby the membrane, as they can expose their hydrophobic components to face it, while in solution these ones are preferentially buried inside to be screened from the solvent \cite{Nguyen2011}.

The folds adopted by AMPs are both $\alpha$-helix or $\beta$-sheet rich structures. Amphiphatic $\alpha$-helices present a charged side which is tailored to face towards the phospholipid head groups and an hydrophobic ones which is favourably buried into the acyl chains core. In the initial phases of the interaction, the peptide lies parallel to the membrane, attracted with its charged side to the membrane, then rearranges to have the two faces in the respective favourable regions; then the helix axis starts to form an angle with the membrane plane, and finally to insert deeper into the lipid core, often spanning the full membrane thickness \cite{Ebenhan2014}.
%
Structures rich in $\beta$-sheets include $\beta$-hairpins, which again show an amphiphatic distribution of residues and, similarly to helices, insert within the membrane after a first flat approach.
%
The final insertion arrangement depends on the peptide characteristics and length, the presence of kinks in its structure (in case of helices), and the interactions with other copies of the peptide.

\paragraph{Membrane disruption} Several models have been proposed to describe the exact mechanisms of AMPs penetration after they bind to the cytoplasmatic membrane, and how their combined action leads to membrane permeabilization (Figure \ref{fig:amp}) \cite{Brogden2005,Toke2005,Nguyen2011}.
%
At low peptide to lipid ratio, the favourable configuration is represented by peptides lying parallel to the membrane plane as described previously \cite{Yang2001}, but an increase in peptide concentration triggers the transition to an inserted state where the main axis of the peptide is perpendicular to the membrane. The organisation of AMPs inside the membrane core can assume different configurations, as described below.

The ``barrel-stave" model proposes that AMPs insert perpendicularly into the bilayer. Recruitment of peptides in the same area results in the formation of a transmembrane pore with a central lumen. The walls of the pore are constituted by the hydrophilic face of the peptides, while their hydrophobic side is interacting with the lipid tails around the pore. This model is adopted for example by the $\alpha$-helical AMP alamethicin, which forms voltage-dependent ion channels by aggregation of four to six molecules \cite{Bertelsen2012,Yang2001,Lee2004,Spaar2004}.
%
In the ``toroidal" pore model instead, the insertion of peptides forces the phospholipid to bend continuously from one leaflet to the other, resulting in a pore defined by both peptides and phospholipids head groups. The toroidal model differs from the barrel-stave model as the peptides are always associated with the lipid head groups even when they are perpendicularly inserted in the lipid  bilayer. The ``toroidal" pore mechanism is induced by $\alpha$-helical magainins, protegrins and melittin \cite{Yang2001,Matsuzaki1996,Hallock2003}, and leads to membrane perturbation which extend further away from the pore, as lipids must rearrange around it \cite{Bertelsen2012}.
%
As a comparison between the two models, alamethicin induced barrel-stave pores have an inner and outer diameters of 1.8 nm and 4.0 nm respectively \cite{Spaar2004,He1995}, while magainin-induced toroidal pores are larger and can vary in their size, with an inner diameter of 3.0–5.0 nm and an outer diameter of 7.0–8.4 nm, involving about 4 to 7 magainin monomers and about 90 lipid molecules \cite{Matsuzaki1998,Matsuzaki1997}.

Finally, in the ``carpet" model, the accumulation of AMPs on the surface of the membrane, laying parallel to it, causes tension in the bilayer. Subsequently the membrane is disrupted by peptides in a detergent-like manner, leading to the formation of micelles \cite{Shai1999,Ladokhin2001}.
%
The critical threshold concentration triggers a cascade effect, in which formation of the first disruption in the membrane allows the penetration of other AMPs in the inner side of the bilayer. The cooperation between peptides on both sides of the lipid membrane enhance the AMP induced curvature on the membrane causing accelerated disruption \cite{Oren1998}.
%
The ``carpet" model mechanism is again observed for peptides presenting an $\alpha$-helical structure, generally with two or more helices connected by short loops (like cecropin \cite{Gazit1995} or ovispirin \cite{Yamaguchi2001}).

The prevalence of examples with an helical structure for the above models derives from the fact that the understanding of how helical AMPs function is often easier than the one of $\beta$-sheet rich structures.
%
Indeed, helices have a well defined fold (at least in the membrane environment), a compact structure, and often a clear segregation of complementary patches that can attract other copies of the peptide and thus promote the self-assembly process necessary for pore formation. On the contrary, many $\beta$-sheet AMPs have a more flexible structure, diversifying their mechanisms of action \cite{??}.

AMPs rich in $\beta$-sheets can be divided into $\beta$-hairpins and peptides from the defensin family \cite{Nguyen2011}.
%
Many representative of the former class disrupt bacterial membranes via formation of toroidal pores: as an example, porcine peptide protegrin I triggers the toroidal pore formation assembling into a $\beta$-barrel structure when in contact with anionic membranes. However, it folds into $\beta$-sheet aggregates on the surface of cholesterol containing membranes, thus acting selectivity on bacterial membranes only \cite{Tang2009}.

Defensins permeabilise the membrane as well but their mechanisms are not as well explored \cite{Lehrer2004,Zasloff2002,Fujii1993}.
%
Some members of the family form transmembrane pores on planar bilayer when a physiologically relevant negative potential is applied to the membrane,\cite{Kagan1990}
%
while others like sapecin from Sarcophaga peregrina form oligomers in phospholipid vesicles \cite{Takeuchi2004}.
%
Although various descriptions of membrane damage have been reported, and include ion channels, transmembrane pores and extended rupture of the membrane, they are likely related, being a modulation of a similar acting principle \cite{Dathe1999}.


\paragraph{Alternative mechanisms of action} Finally, many other non-lytic mechanisms are suggested for $\beta$-sheet AMPs: defensin A from P. terramovae reduces the cytoplasmic potassium concentration, partially depolarising the inner membrane; tachyplesin from horseshoe crabs is able to bind to the minor groove of DNA, interfering DNA–protein interactions \cite{Yonezawa1992},
%
and bovine lactoferricin can act synergistically with other antimicrobial agents by affecting the transmembrane potential and proton-motive force, resulting in inhibition of ATP-dependent multi-drug efflux pumps \cite{Gifford2005}.
%
Moreover, after translocation within the cell, bovine lactoferricin can also inhibit DNA, RNA and protein synthesis. Section \ref{sec:capzip} will treat in detail the functioning of lactoferricin distinguishing its role as membrane active peptide versus intra-cellular targeting compound: indeed, many works have focussed on its antimicrobial processes and on locating the section of the sequence performing the membrane disruption activity \cite{Tomita1994,Hwang1998,Schibli1999,Nguyen2005} to understand whether it retains the efficacy regardless of the fold. These investigations, together with similar ones conducted on other AMPs \cite{???}, provided the discovery of first minimal functioning antimicrobial blocks, promoting the understanding of how AMPs work in general and boosting the design of tailored AMPs from specific sequences.


\subsection{Mechanisms of resistance to AMPs}

Antimicrobial peptides were introduced in this review as a class of new drugs and a possible solution to the crisis of antimicrobial resistance. Any new drug entering the pool of the clinically approved compounds is (at least temporary) a solution to the problem of resistance to known antibiotics, but it must be clarified that bacteria can develop resistance to AMPs too.
%
As such, AMPs are not a definitive solution to the problem; never the less, the resistance to their action is generally not based on dedicated resistance genes that are conferred by horizontal gene transfer, as in the case of many antibiotics resistance mechanisms \cite{Peschel2006,Juhas2015}.
Because of that, a certain increase of resistance after exposure to the drug is to be expected (`MIC creep’), %such as observed for the AMP-related antibiotic, daptomycin [126,176]
but it is less likely to spread quickly to other species.

Some of the mechanisms of resistance to AMPs are similar to the ones employed by bacteria to counteract small molecule drugs, for example overexpression of efflux pumps to dispose the AMP, proteolytic degradation of the peptide by extracellular enzymes, or sequestration by the bacterial biofilm matrix preventing accession to the target; while others tackle the specific action of AMPs on the cell membrane, and prevent it by modification of the surface or the cytoplasmic membrane. Table \ref{table:AMP_res}, from Ref. \cite{Joo2016} gathers these mechanisms offering examples for each of them in both Gram-positive and Gram-negative bacteria.
%
In the following paragraphs some of them are explained in more details, with the omission of efflux pump, for which the principle is very similar to what explained in Section \ref{sec:AMR_mechs} for the resistance to small molecules antibiotics.

\begin{figure}[h]
\footnotesize
\vspace{0.5cm}
\centering
 \def\arraystretch{1.1}
\begin{tabular}{lll}
\hline \\
\textbf{Mechanism} & \textbf{Gram-positive bacteria} & \textbf{Gram-negative bacteria} \vspace{0.3cm}\\
 \hline \\
Extracellular proteins & \specialcell[t]{Proteolytic degradation\\Sequestration} &  Proteolytic degradation \vspace{0.35cm} \\
Exopolymers & PIA, PGA & Alginate, polysialic acid \vspace{0.35cm} \\
Surface modification & \specialcell[t]{Repulsion\\$\ \ \ $(D-alanylation of TA)\\Steric hindrance \\$\ \ \ $(L-rhamnosylation of WTA)\\Lipid II modification} & \specialcell[t]{Repulsion (lipid A\\$\ \ \ $phosphate modification)\\Increased OM rigidity \\$\ \ \ $(lipid A acylation)\\O-antigen of LPS} \vspace{0.35cm} \\
\specialcell[t]{Cytoplasmic membrane\\alteration} & \specialcell[t]{Charge repulsion \\$\ \ \ $(PG amino-acylation)} & \specialcell[t]{Increased IM rigidity\\$\ \ \ $(PG acylation)} \vspace{0.3cm} \\
%Efflux pumps & $\circ$ Export by ABC transporters & \specialcell[t]{$\circ$ Export by RND family\\$\ \, $ efflux pumps} \vspace{0.35cm} \\
\hline
 \end{tabular}
\captionof{table}[Overview of bacterial resistance mechanisms against AMPs]{Overview of bacterial resistance mechanisms against antimicrobial peptides. Adapted from Ref. \cite{Joo2016}}
\label{table:AMP_res}
\end{figure}

\paragraph{Proteolitic degradation and sequestration [TAKE OFF?]}
The first defence of bacteria against AMPs are the proteins secreted on the extracellular side of the membrane, as some of them, the proteases, are able to degrade peptides and thus AMPs.
%
For example, staphylococci secrete various metalloproteases such as aureolysin and SepA, and serine endopeptidases such as the V8 protease, which are known to degrade linear AMPs as the human cathelicidin LL-37 \cite{Sieprawska-Lupa2004,Teufel1993}; while group A Streptococcus produces a cysteine protease able to disrupt many host AMPs, including LL-37 and beta-defensins \cite{Schmidtchen2002,Baranska-Rybak2006,Nelson2011,Frick2011}.
%
In general linear AMPs are more easily degraded than the ones with non-linear structures containing disulfide bonds\cite{Peschel2006} such as defensins \cite{Selsted1989}.
However, some bacteria have evolved proteases able to degrade even these AMPs with increased stability; for example protein OmpT from the omptin family contributes to resistance in E. coli being able to degrade the AMP protamine \cite{Stumpe1998} which is thought to have a nonlinear structure involving three disulfide bonds \cite{Biegeleisen2006}.
%
In turn AMPs can be shielded from the action of proteases by binding to proteins such as extracellular actin, preventing the access of degradative proteases while still maintaining the activity \cite{Sol2014}.

Other more complex mechanisms of degradation are possible, like the ones occurring in the intracellular environment after the AMP is being imported by specific transport proteins \cite{Groisman1992,Parra-Lopez1993,Mason2005} or the ones related to host immune response related proteins with AMP-degrading activity \cite{Taggart2003}.

Another process relevant for the neutralisation of AMPs at the extracellular environment level is the sequestration of the peptides: as an example, staphylokinase is one of the most prominent extracellular AMP-sequestering molecules \cite{Bokarewa2004,Jin2004}, which inactivates $\alpha$-defensin binding to them and preventing their interaction with the designed target.


\paragraph{Biofilms}

Bacteria can resist AMPs by organising into specialized structures known as biofilms. These structures are formed by sessile bacteria adhering to a surface in organized manner that allows the circulation of nutrients \cite{Costerton1999}.
%
Bacteria in a biofilm secretes an extracellular matrix with adhesion and protection functions. This matrix includes various compounds as cellulose, teichoic acids, proteins, lipids and nucleic acids \cite{Jolivet-Gougeon2014} and confers resistance to antibiotics and AMPs, in some cases 1000 times as great as the one developed in their planktonic form \cite{Nickel1985,Mah2001}. This is achieved by repulsion and/or capture of AMPs by mainly exopolysaccharid or capsular polysaccharides.

For example polysaccharide intercellular adhesin (PIA) produced by S. aureus and a variety of other bacteria is responsible for the resistance to both cationic AMPs (HBD-3, LL-37) and anionic dermcidin \cite{Wang2004,Vuong2004PIA}: this is done by deacetylation of PIA, which increases its positive net charge, thus repelling more efficiently cationic CAMPs, increasing sequestration of dermicidin, as well as forming a mechanical barrier for both of them \cite{Vuong2004}.
%
Similarly, in other cases it is structural hindrance as well as electrostatic trapping that prevents cationic AMPs to penetrate the biofilm (as in observed for polymyxin B, HNP-1,HBD-1, lactoferrin and protamine on K. pneumoniae, S. pneumoniae or P. aeruginosa) \cite{Campos2004,Llobet2008}.

Finally, it is important to note that AMPs are particularly promising as alternatives to traditional antibiotics in the treatment of biofilm-associated infections. Indeed in this type of infections (where bacteria are growing slowly) it is advantageous to have bactericidal agents as opposed to bacteriostatic ones targeting fast-growing bacteria, as the majority of traditional antibiotics \cite{Batoni2011,Strempel2014}.
%
Therefore, biofilm-intrinsic AMP resistance constitutes a great challenge a field already depleted of efficient cures \cite{Joo2012,DiLuca2014}.

\paragraph{Surface remodelling}
As mentioned in the previous two paragraphs, the bacterial cell envelope environment constitutes a major impediment for AMPs activity.
%
Even if a peptide reaches the bacterial envelope intact, bacteria can modify the characteristics of their surface to prevent its efficient action.
%
Gram-positive and Gram-negative bacteria put in place different strategies to do that, according to their distinct cell envelopes. In particular, the target of such modifications are the teichoic acids (TA) in Gram-positive cell wall and lipopolysaccharide (LPS) in the Gram-negative outer membrane.
%
For example, D-Alanylation of TA, observed in Staphylococcus, adds a positive charge to it, reducing the attraction of cationic AMPs including protegrins \cite{Peschel1999,Fabretti2006,Saar-Dover2012}.
%
This in turn increases the cell wall density, reducing the surface permeability \cite{Saar-Dover2012}.
%
For Gram-negative bacteria (like P. aeruginosa), the LPS positive charge is increased by addition of different amine-containing molecules \cite{Moskowitz2004,Gunn1998} or by removing phosphate lipids, which have a negative charge, from lipid A, one of the constitutive moieties of LPS \cite{Wang2004MsbA,Wang2006}.

Another target of AMPs in Gram-positive bacteria is the bacterial peptidoglycan precursor, lipid II, which has a key role in the formation of the cell wall, thus many bacteria started using a modified version of it.
% substantial change!
The best known case is the replacement of its terminal D-alanine with D-lactate or D-serine \cite{Bugg1991} to avoid the action of the glycopeptide vancomycin, which works binding to the D-Ala-D-Ala terminal moieties of the precurson, preventing cross linking of molecules between them and thus the cell wall synthesis \cite{Brotz1998}.

Gram-negative bacteria instead can enhance the rigidity of the outer membrane to reduce permeability to AMPs via addition of extra acyl chains into lipid A \cite{Guo1998,Bishop2000}. The long polysaccharide chain of LPS (called O-antigen) makes this class of bacteria particularly resilient to the action of AMPs \cite{Silhavy2010} as both the LPS core and the O-antigen were proven to promote AMP resistance in B. cenocepacia and Brucella abortus using mutants that lack the respective sugar structures in LPS \cite{Loutet2006,Allen1998}.

Surface modification to counteract the AMPs activity occur very often also in the cytoplasmic membrane as it is the final target of many antimicrobial peptides. In the eventuality that the AMP successfully passes the cell wall and reaches this membrane, it is attracted to its surface by the negative charge of the lipids composing it. In particular, the cytoplasmic membrane is rich in phosphatidyl-glycerol (PG) and diphosphatidylglycerol (DPG, also called cardiolipin) lipids: their negative charge can be masked by amino-acylation of the PG head group, so that the final compound repels AMPs through electrostatic interaction \cite{Peschel2001}. Usually the group added is a Lysine \cite{Thedieck2006}, but Alanine is commonly chosen as well \cite{Klein2009}.

The rigidity of the cytoplasmic membrane can be enhanced as well by an increase of saturated acyl chains and has been proven to confer resistance \cite{Band2014,Kumariya2015}, though the precise mechanisms underlying the connections are still partially unclear.

\hspace{0.5cm}
\\
Finally, often resistant bacteria employ many of the aforementioned strategies at the same time, for example the modification of the surface charge together with the modification of other membrane components for a decreased recognition and augmented rigidity \cite{Raetz2007}.


\subsection{Principles of AMP design} \label{sec:amp_design}

It was already highlighted in Section \ref{AMP_mechs} that the classification of AMPs provides knowledge on which characteristics a sequence must have to perform an antimicrobial function.
%
At the present state of the art, several databases exist gathering AMPs and subclasses of them, like membrane active, biofilm active or haemolytic peptides \cite{APD3, ???}. Based on the increasing amount of data, it is now possible to identify five features which, comprehensively, discriminate AMPs with respect to non antimicrobial peptides:
%
\begin{itemize}
\item \textbf{Structure}: as mentioned before, both $\alpha$-helical and $\beta$-sheet rich AMPs exist, as well as mixed structures. Short helix ($\sim$ 22 amino acids) \cite{???} and short $\beta$-sheet ($\sim$ 10 amino acids) \cite{???} are particularly common and the structural difference is reflected in the slightly different mechanisms of actions (when known). When screening a peptide to identify its AMP-likeness, it must be considered that some sequences may rearrange in proximity of the membrane, thus their structure in solution does not reflect their active conformation.
\item \textbf{Charge}: AMPs are charged moieties. Usually they present positive charge (up to $\sim + 10\,e$), but there are examples of anionic ones \cite{???see_example_on_AMPR}. Their potency is related to the amount of charge each unit possesses, however an increased charge may promote haemolytic activity as well. \cite{???}
\item \textbf{Hydrophobicity}: together with charged amino acids, AMPs contain also hydrophobic residues, usually with abundance of aromatic chains and specifically Tryptophan. Indeed membrane active peptides must insert into the lipid core of membranes, which is an hydrophobic environment, therefore having such residues help them in anchoring in such region.
\item \textbf{Amphipathicity}: to host both the charged and hydrophobic residues, most AMPs organise themselves in an amphiphatic structure, i.e. the two types of amino acids side chains are located on the opposite side of the peptide. The usefulness of this segregation in the anchoring and pore formation mechanism has been explained in Section \ref{sec:AMR_mechs}.
\item \textbf{Solubility} AMPs need good solubility to prevent aggregation in the aqueous environment they float in before arriving to the membrane, as aggregation would most likely impede their optimal interaction with the membrane.
\item \textbf{Sequence motifs}: finally, a long debate exist on whether the effectiveness of AMPs is related to particular sequence motifs or only to the overall amino acid composition. Statistical methods are trying to extract relevant pattern from the databases available, however the details of the structure-activity relationship are still uncertain.
\end{itemize}
%
Based on the above features, it is possible to make prediction on whether an amino acid sequence is antimicrobial or not. Several online servers (APD3, dbAMP, DBAASP, antiBP2, amPEP \cite
{APD3,DBAASP2,dbAMP,antiBP2,amPEP}) are available to host curated databases and, based on statistical or machine learning methods elaborated on those, evaluate the antimicrobial properties of user provided sequences: the output is generally a score of how likely the peptide is to have such function, with sometimes an optional indication on whether the activity might be stronger against Gram-positive or Gram-negative bacteria.

At present, it is still impossible to foresee the precise efficacy and mechanism of action of an AMP from its sequence only (while more information can be derived if a structure is present as well). 
%
The knowledge of structure-activity relationships for AMPs would be beneficial to find new, better performing ones. Indeed the design of new AMP sequences aims at improving some specific characteristics:
\begin{itemize}
\item \textbf{specificity} against particular bacterial species;
\item \textbf{stability} against the action of proteases, thus allowing a longer residence time in the body;
\item \textbf{low cytotoxicity} at the therapeutic dose required (so an high therapeutic index).
\end{itemize}
The need for such improved peptides lies in the fact that their natural form constitutes the first broad spectrum defence our body employs against infectious bacteria and thus are often of mild potency. However, foreseeing their application as future drugs, it is desirable to tailor them to fulfil different criteria according to the infection to treat.
%
Several methodological approaches to AMP design are possible, and they can be grouped in three main lines: template based studies, biophysical studies and virtual screening \cite{Fjell2011}.


\paragraph{Template based studies}
The main idea behind template based methods consists in employing existing antimicrobial sequences and modifying them in the direction of more potency or less toxicity. The most widely explored templates are cecropin, magainin and protegrin for their short sequences and because their action and structure has been well characterised \cite{Wiradharma2011,Huang2010,Pag2008,Wang2015}.

Ideally, an amino acid scanning of all the residues in an AMP provides information on the role of each of them thus prompting at the most suitable mutations. High-throughput methods allow nowadays for such thorough investigation in the case of short AMPs \cite{Hilpert2005,Hilpert2006}; similarly, a less resource consuming Alanine scanning points at the most critical residues on which all the mutations can be tested \cite{Migon2018,Grieco2011,Xie2018}.
%
In the absence of such resources, earlier studies focused on simpler approaches to enhance charge and amphiphilicity of the peptides, as these characteristics are deemed crucial in their effectiveness (see the paragraph above) \cite{Wiradharma2011,Huang2010,Pag2008,Wang2015}.
%
Similarly, the addition of acyl moieties have been shown to improve the performances of AMPs, as these can provide the necessary hydrophobic domains that, together with charged amino acids, allow an amphiphatic structure in short pep­tides \cite{Radzishevsky2005,Serrano2009,Avrahami2004}.

The above methods however focus on the single amino acid level and can not take into account the interplay between two residues, while the paired mutation of two of them can give optimal results with respect to a single intervention.
%
Furthermore there is little to no information on the three dimensional structure of the mutated peptide. Without such information, it is difficult to extract general rules on why some mutations work better, and often the results of these studies give indeed enhanced AMP, but cannot be generalised to other sequences.

Only recently a structure based approach has been developed to integrate structural information on template based models to design peptides active against many bacterial lines at the same time \cite{Liu2018}.
%
Similarly, an example of template design combining chemical and case specific structural information comes from Jiang et al. \cite{Jiang2011} who designed AMPs with improved selectivity for bacterial membranes. Starting from a synthetic broad spectrum AMP with high toxicity, the positioning of positively charged residues at the centre of the non ­polar face of the amphipathic $\alpha$-helix reduced its haemolytic activity while improving its therapeutic index. This proves that charge and structure features do affect the antimicrobial activity, but again does not provide very generalised design rules.

A complementary approach consists in focussing on minimal antimicrobial blocks: several investigations proved the importance of single residues and their intercalated pattern in natural and designed AMPs. In particular, natural AMPs are rich in Tryptophan and Arginine residues \cite{Chan2006}, while synthetic ones have been produced with only Lysines in combination with Leucine, or Arginine with Valine to produce amphi­pathic helices \cite{Deslouches2005}. Furthermore, polyarginine are long known for being cell penetrating peptides \cite{Schmidt2010}.

An effort to extract principles from these examples is represented by text based model where amino acids constitutes the letters and patterns occurring in natural AMPs are the grammar rules \cite{Loose2006}.
This approach can benefit of the improving size of peptide database, together with the advancement in text mining technology and dedicated machine learning algorithms, bringing to the streamlined selection of promising sequence to investigate further \cite{Cipcigan2018}.

In general, the advantage of template based methods is in the reduced number of sequences to test, with decreased cost, as only a subspace of them is explored, namely the ones close to the original template.


\paragraph{Biophysical studies}
Biophysical studies aim at understanding the functioning of AMPs by investigating their structure. Free energy perturbation, molecular dynamics (MD) simulations and thermodynamics calculations can all provide knowledge on how the three dimensional arrangement of residues is important to allow their functional role.
%
Contrary to sequence based methods, these techniques give an insight into the mechanism of action of an AMP: free energy perturbation allows for example to pinpoint the interactions that stabilise a structure, while molecular dynamics simulations can show why a particular residue is favourably binding to the membrane. Some mechanistic hypotheses can be deduced from the analysis of many AMPs sequences, but it is only zooming into the details that these hypotheses can be confirmed. Moreover, structural information is crucial to discriminate cases in which similar sequence stretches behave differently due to the environment around them, as this information is necessarily lost in a sequence only approach.

The drawbacks of such techniques lay in their computational cost. All of them can approach systems with a limited size, and simulations can access short (microseconds) time scales preventing the reproduction of phenomena of the order of millisecond (a detailed overview of the state of the art, advantages and drawback of MD simulations will be given in Chapter \ref{chapter:MD}).
%
For these reasons, such techniques have been applied to fewer systems in comparison to sequence based screenings, and only few mutations have been tested and compared in silico.

As such, the strength of biophysical studies does not lay in the power of analysing large dataset, but rather in the fact that, as they exploit the whole information available (sequence, structure, chemistry), they can single out the interactions that are crucial for a mechanism, clarifying whether they are peculiar of a given local environment or they can be transferred elsewhere. In this respect, they provide a generalisable knowledge applicable to different systems and thus to the design of novel AMPs at the atomistic level.

An example of how MD simulations shed light on AMP-membrane interactions is given by the protegrine peptide: porcine protegrin is a $\beta$-hairpin AMP which is though to act through pore formation. A model for the detailed steps of such mechanism was obtained by simulations and proposed a non trivial process of electrostatic attraction to the anionic membrane, followed by dimeri­sation and subsequent insertion into the membrane. Finally, inserted peptides form large aggregates that lead to transmem­brane pores formation \cite{Bolintineanu2011}.
%
Further steps were taken for ovispirin \cite{Khandelia2005}, indolicidin \cite{Tsai2009} and temporin \cite{Farrotti2017} sequences, designing and testing, first computationally and then experimentally, mutants of the original sequence with improved activity and decreased haemolytic activity.

\paragraph{Virtual screening}
Contrary to biophysical assay, virtual screening methods are employed to analyse a large number of sequence, when an experimental or computational test of all of them would be prohibitive. The key concept of these methods consists in the identifications of some descriptors which allow to predict the potency of the sequence: from the analysis of a database of AMP with known activity, a model is created and used to score novel synthetic sequences. The aforementioned text-based methods fall in this category, but they can be generalised to include more information than the sequence composition only.

These methods are witnessing new popularity due to the recent evolution of machine learning (ML) techniques: if originally they relied on regression methods, in the past three decades artificial neural network have been extensively applied to the problem (for a historically informed review see Table 1 in Ref. \cite{Fjell2011}). Machine learning appears particularly suitable to the task as the potency of AMPs is certainly determined by the combination of many factors, and it is difficult to properly weight them and identify the predominant ones in each context.

Therefore ML algorithm are trained on a set of AMPs sequences labelled by their potency, where the sequence (or each amino acids) can be characterised by many different properties: partial charge, hydrophobicity, amphiphilicity and molecular weight are the most intuitive ones; but also experimental measures of pK (the logarithm of the dissociation constant), pI (the isoelectric point), nuclear magnetic resonance data, chromatographic indices retention time in a given chromatography column, octanol–water partition fraction or circular dichroism data. Finally, other theoretically computed features as van der Waals surface area and hardness (the energy required to remove the outermost electron) can be considered as well.
%
The more the input features to consider, the more expensive is to train the model, but an higher accuracy can be meet: the power of such approach is exactly in the identification of relevant features traditionally overlooked. At the same time though, the output is likely providing complex descriptors (i.e. combinations of many features) of difficult interpretation. This is why the step of features selection is important and more than one model is trained to identify the minimum set which gives satisfactory agreement with the data \cite{Walters2005,Gonzalez2008}.
%
In principle, having complex descriptors is not a problem as, rather than guide the design, they can be used to scores all possible combinatorial sequences of the desired length to identify the best ones. In practise this can be difficult even for such an in silico screening because of their number (as a meter, all the possible amino acid combinations on a 10 residues sequence are of the order of $10^{11}$).
%
A possible solution is constituted by evolutionary search methods in the sequence space: analogously to an energy minimization process in space, single amino acid mutations are attempted and the fitness of the new sequence computed, based on the model generated via ML. The move is accepted if it proceeds toward an improved fitness, i.e. a maximum in the fitness landscape.

Another obstacle to ML procedures is given by the fact that the more features one wants to consider, the more sequences need to be given as input to the algorithm: nowadays, high-throughput synthesis methods, together with surrogate measures of bacterial killing (such as lipid vesicle experiments \cite{Wimley2010} or the diminished energy ­dependent luminescence of bacteria constitutively expressing luciferase 90, rather that Minimum Inhibitory Concentration assays \cite{Lewenza2005}), allow for quick screening of many of those.
%
This procedure was employed by Cherkasov et al. \cite{Cherkasov2009}, as they assessed the antimicrobial properties of thousands of 9 residues sequences and trained a neural network on the outcome, to then score novel sequences with good accuracy as proved by experiments.

More recently, new studies are exploring novel machine learning approaches, with the aim of selecting sequences for experimental assays \cite{Lee2017}, while some work suggests the combined use of machine learning and molecular dynamics for an improved selection of interesting peptides.

\subsection{Clinical applications}
Antimicrobial peptides have been studies for many year, however the push to capitalise them to get compounds viable for the clinical stage has been delayed by many factors, including production costs, lack of interest in the face of more potent small molecules which were deemed more economically advantageous by pharmaceutical companies.
%
The constant creeping of AM resistance though has focussed more effort on this class of compounds, mainly from small biopharmaceutical companies, and at present several of them are in clinical trials, in phase 1 or 2 \cite{Naafs2018}.

The two major problems encountered so far for the AMPs sequences in trial are their liability to proteolytic degradation, and their unknown toxicology profile when administered systemically \cite{Hancock2006}. For the last reason in particular, many of them in are trial for topical use against skin infection only, while they are deemed unsuitable for internal administration.
%
Design of novel AMPs can be tailored to improve the liability to degradation, for example introducing D-amino acids, non ­natural amino ­acid analogues of opposite chirality, which, with appropriate formulations, are mimetic to the immune system \cite{Wipf2009,Choudhary2011}. Machine Learning protocols can help in pre-screening their toxicity through virtual screening methods.

Overall, antimicrobial peptides remains a promising tool to counteract infections and, as their design is still - comparatively - in its infancy, there is room to explore novel applications and synthesise improved sequences apt to get to the clinical stage.


\section{Gene therapy}

% REFS
% Anguela2019, Barrangou2015, Hsu2014, Lawler2017, Mingozzi2011, Naldini2011, Oldfield1993, Zhang2014cas

Finally, we want to bring to the reader's attention another class of therapy for non infectious diseases which evolved in the last decades and is non the less relevant for the work of this thesis: gene therapy. Briefly mentioned in Section \ref{sec:organic} when introducing Adeno Associated Viruses, in recent years gene therapy has greatly evolved and gained attention for the treatment of tumours, genetic diseases and complex acquired disorder \cite{Anguela2019}.

The key concept is the delivery of genetic material to sick cells which possess a faulty copy of a gene, to influence its expression. Such fault can result in lack of synthesis of the protein of interest or in its misfold and/or misfunction. The correction can be performed in three different ways: augmentation gene therapy introduces an healthy gene copy to restore the normal functionalities of the protein of interest and thus of the cell; suppression gene therapy suppresses a detrimental gene (this is particularly useful in the case of cancer, where the goal of the therapy is to impede cancer cell replication); gene-editing, the most recent advance in the field, overlooks the possibility of correcting base pairs mutations to restore the original healthy sequence (Figure \ref{fig:gene_therapy}).

\begin{figure}
\begin{center}
\includegraphics[width = 0.9\textwidth]{pics/gene_therapy.jpeg}
\caption[Principles of gene therapy]{Principles of gene therapy. Reproduced from \cite{Anguela2019}.} \label{fig:gene_therapy}
\end{center}
\end{figure}

For the first strategy mentioned above, the therapy itself usually consists in the delivery of a DNA strand, which in turn can be internalised in the genome and thus spread when the cell replicates, or not internalised and thus influence the functionalities of that particular cell only. The first approach is used mainly for ex vivo administration (in cultured cells taken from the patient that are subsequently transplanted back), while the second for in vivo one (directly into the patient).
%
The second strategy, gene repression, employs RNA interference with in vivo therapy, and therefore aims at delivering miRNA or siRNA strands which repress the transcription of the problematic RNA sequence.
%
Finally, gene-editing is often done through the functionalisation of the CRISPR-Cas9 technology. This mechanism is found in prokaryotic organism as bacteria and archea as defence against viruses \cite{Barrangou2015}.
CRISPS (clustered regularly interspaced short palindromic repeats) is a library of DNA fragments from viruses that have previously infected the prokaryote. The Cas9 enzyme (``CRISPR-associated protein 9") uses these sequences to recognize and cleave strands of DNA complementary to the CRISPR sequence. In doing so, it blocks the reproduction of viruses if a following infection occurs. Research has been done to engineer the CRISPR-Cas9 technology to edit (rather than simply cleave) genes within organisms \cite{Zhang2014cas,Hsu2014} and thus perform a therapeutic role.
%
More complex strategies are possible combining gene and drug therapy, such as the delivery of suicide gene increasing the sensitivity of tumour cells to cytotoxic drugs \cite{Oldfield1993}, or the use of oncolytic viruses (OVs) that selectively replicate in tumour cells (disrupting them) without harming normal cells \cite{Lawler2017}.

The main challenge in the development of such therapies lies in the identification of a suitable vector: delivery of free genome in solution results in poor internalisation and therefore low therapeutic effect. Vectors allow the DNA/RNA to enter effectively into the cell: viruses can be used, modifying their genome to include the necessary sequence and remove the ones promoting viral replication.
%
According to the administration type (ex or in vivo), integrating and non integrating viral species are used, i.e. which have a mechanism to transfer their genome into the host one or not (specifically the most used ones are lentiviral and adeno-associated viral (AAV) vectors respectively \cite{Naldini2011,Mingozzi2011}).

Despite the challenge posed by such a genome modification tool, and the risk associated (for example deleterious insertional mutagenesis or deleterious immune responses), at present six gene therapies have received approval in the Western world \cite{Anguela2019} with many more undergoing regulatory review. 

The outlook of gene therapy research nowadays lies not only in improving the cargos to cure at the molecular lever more diseases, but also in the research of appropriate vectors with low toxicity, low induced immune response and high delivery efficiency. In that respect synthetic vectors start to be investigated for a virus-free delivery strategy. The system studied in this thesis proposes indeed, among its other function, to delivery genetic material into human cells.


\section{Closing the circle: an antimicrobial drug delivery vehicle}

% REF
% Aguilera1999, Arnold1977, Arnold1980, Arseneault2010, Bellamy1992, Bustin2009, Castelletto2016, Chan2006, Cochran2001, Crombez2009, Elsabahy2013, Farnaud2003, Gahr1991, Gifford2005, Hwang1998, Jahani2015, Kirkpatrick1971, Nguyen2005, Sanchez1992, Shau1992, Schibli1999, Strom2002, Tomita1994, Tsutsumi2012, Wakabayashi1996

Twice in this introduction peptide design has been brought to the reader's attention. First, design can engineer self-assembling building blocks for the formation of delivery scaffolds. Second, it can produce antimicrobial peptides with improved potency or selectivity, and/or reduced toxicity. As design is not bound to natural rules, it can foresee and imagine multifunctional materials which are not observed in nature. In particular, the introduction above poses the question of whether it is possible to engineer peptides able to perform both an antimicrobial and delivery functions at once.
%Thus, the relevance of such compound would be twofold.

Such self-assembling antimicrobial compounds would have a twofold interest for medical applications.
%
First of all, self-assembly is functional to the antimicrobial activity: many AMP sequences have a weak potency, and only a high (critical) concentration can trigger the bactericidal mechanism. This is intuitive in the case of the carpet model strategy (see Section \ref{AMP_mechs}), where AMPs lay homogeneously on the surface of the bacterial membrane and breaks it upon sufficient coverage of the area. Also the barrel-stave and toroidal pore models rely on the mutual interaction between peptides to maintain the pore edges.
%
Generally, as AMPs are positively charged, the presence of many copies of a sequence enhances the local electric field and charge imbalance, which are critical to the membrane stability. Thus a self-assembling sequence will enhance the local concentration which is crucial to initiate the membrane disruption.
%
Second, in order for the assembly to be able to perform the additional delivery function, it must be able to either organise in a tailored structure (for example a capsule able to host a drug), rather than an amorphous aggregate, or to co-assemble with the cargo of interest.

Out of all the possible applications, the most promising is perhaps the use of such vehicles to deliver drugs to treat metabolic or genetic diseases: while the cargo tackles a defect of the host system, the vehicle can counteract the proliferation of bacteria. This is particularly important in situation where the host immune response is weakened and thus infections normally harmless can spread and cause damage.
%
At this point it must be noticed that the cargo is not bound to be a small molecule: many organic biomolecules proved to have medical potential, and as long as they can effectively co-assemble with the carrier the whole system makes a suitable synergy. As mentioned in the previous section, gene therapy is also an actively expanding field which looks with interested at the development of carrier for genetic material. Given that viruses have been the first choice for DNA/RNA delivery so far, peptidic carriers seem a natural evolution of them.

\bigskip
Given the above premises, this thesis aims at elucidating the behaviour of a particular synthetic self-assembling peptide which is suitable for antimicrobial activity and gene delivery: characterising it will complete the knowledge on its mechanisms and will complement the broader information already known on such functional building blocks. This will be crucial for the engineering of new ones with improved characteristics, either regard their antimicrobial activity, assembly performances, or tailored cargo delivery.


\subsection{Capzip} \label{sec:capzip}

\begin{figure}
\begin{center}
\includegraphics[width = \textwidth]{pics/capzip.png}
\caption[Cazip molecule]{Capzip molecule scheme and bond representation. [TO BE IMPROVED] Adapted from \cite{Castelletto2016}.} \label{fig:capzip}
\end{center}
\end{figure}

The molecule capzip has been designed to perform the functions mentioned above. To recapitulate, the properties the designed block must have are:
\begin{enumerate}
\item assembly into nanoscale virus-like capsules with and without nucleic acids, to ensure the vector is autonomous and thus there is flexibility in the choice of the cargo;
\item antimicrobial activity of the molecule and of the capsule on a time scale useful for therapeutic application;
\item promotion of gene transfer into mammalian cells when co-assembled with the designed peptide, without causing cytotoxic and haemolytic effects.
\end{enumerate}
%
Furthermore, the design effort was aimed at building a template structure of minimal complexity, in order to synthesise a short sequence. Arguably, short sequences are also more flexible in their assembly: it is thus important to explore them and to prove whether even small blocks can in principle form ordered structures.

Based on the above requirements, two design principles emerged: first the employment of a non-linear structure. There are indeed some evidence suggesting that non linear peptides are more prone to assemble in three dimensional structures, opposed to planar ones \cite{???}, and this holds in particular for short sequences which do not fold into a defined secondary structure.
%
The second principle consists in the use of a template antimicrobial sequence which is short and has proved potency.
%
To satisfy the above guidelines, a short peptidic scaffold constituted by a $\beta$-Alanine and two Lysins has been engineered.
%
Three identical copies of the antimicrobial sequence are covalently bonded to the N-terminus of the scaffold sequence and to the nitrogen atom of the Lysin residues side chains (Figure \ref{fig:capzip}).
%
The antimicrobial sequence of choice has been derived from the antimicrobial peptide bovine lactoferricin, which is in turn a portion of the Lactoferrin protein.

\paragraph{Lactoferrin} Lactoferrin is an iron binding protein present in milk (in which it is most abundant, hence its name), saliva and other secretions, as well as in polymorphonuclear leukocytes.
%
It works as an iron binder and provides a natural defence against bacteria and fungi \cite{Sanchez1992,Arnold1977,Arnold1980,Kirkpatrick1971,Jahani2015}, constituting a first defence for infants.

Lactoferrin contributes to bacterial suppression in several ways. At present, its known modes of action falls in three categories: first thanks to its iron sequestering capabilities, it removes essential substrate required for their growth \cite{Farnaud2003}; second it interacts with bacterial membranes and in particular binds to lipopolysaccharides of bacterial walls, oxidising them and affecting the membrane permeability with consequent cell lysis \cite{Farnaud2003}; finally it is implicated in the stimulation of different immunological cells (killer cells \cite{Shau1992}, polymorphonuclear leukocytes, and macrophages \cite{Gahr1991}).
The peptide fragment responsible for binding to the bacterial membrane, named lactoferricin (Lfcin), has been identified near its N-terminus and found to have a more potent bactericidal effect than the intact lactoferrin on a wide range of bacteria \cite{Gifford2005,Bellamy1992,Tomita1994,Wakabayashi1996}.
%
Similarly, a synthetic short peptide derived from a subsequence of human lactoferricin has been proven effective against bacteria as it depolarises the cytoplasmic membrane decreasing the pH gradient \cite{Aguilera1999}.

The bovine homolog of lactoferricin (LfcinB) has a higher bactericidal potency than human lactoferricin on several bacteria \cite{Cochran2001} and therefore has been more extensively studied. Its active core LfcinB is a 25-amino acid sequence which adopts a helical conformation in the full structure but when isolated crystallises in a $\beta$-hairpin with a disulfite bridge nearby the terminals stabilising the fold. An analogous bond is present also in the intact lactoferrin, but was shown to be not essential for bactericidal activity \cite{Cochran2001}.
%
Further experiments on LfcinB subsequences identified an even shorter antimicrobial core, constituted by the six amino acids RRWQWR \cite{Schibli1999}. This core presents a characteristic Tryptophan zipper motif WTW, which in nature very often appears in $\beta$-turn and $\beta$-sheet conformations paired to other copies of the same motif, with Tryptophan rings from facing strands packed tightly against one another in an alternated way \cite{Cochran2001}.
In general, the 6 amino acid sequence contains both charged and hydrophobic residues, in line with the composition of antimicrobial peptides. Therefore, its antimicrobial action is likely derived from its interaction with biological membranes through charge recognition first and aromatic rings insertion in a second moment, however its precise mechanism of action is still unknown.

To elucidate that, several experimental investigations have been carried on LfcinB and its subsequences. First the structure of LfcinB in solution has been investigated by NMR, resulting very flexible \cite{Hwang1998}. Then the binding of its antimicrobial core to sodium dodecyl sulfate micelles was studied \cite{Schibli1999}, which seems favoured by the interaction of aromatic residues with the micelles surface.
%
Similar experiments on large unilamellar vesicles constituted by lipids modelling biological membranes (i.e. ePE:ePC for a model of a mammal membrane, and ePE:ePG or ePC:ePG for a bacterial membrane model)  showed a preferential binding to the latter ones, based on Tryptophan fluorescence \cite{Nguyen2005}. This suggests a selective antimicrobial action on anionic membranes.
%
More experiments have been performed on the full sequence or subsequences hosting some mutations \cite{Tsutsumi2012,Arseneault2010} to investigate the binding with different membrane models: as the systems investigated are slightly different, as well as the experimental conditions, it is difficult to relate them and give a unified interpretation of the modes of action of LFC derived peptides.

Finally, an alanine scanning has attempted to clarify the role of each amino acid in the antimicrobial activity of the peptide \cite{Strom2002}, suggesting a binding role for the Tryptophan residues. This is in line with one of the roles Tryptophan can assume in antimicrobial peptides \cite{Chan2006}, despite other ones have been highlighted, as its propensity to form hydrogen bonds results in some cases in its positioning at the interface between solution and membrane, rather than inside the latter (which happens instead when Tryptophan residues have a binding role).

\paragraph{The designed block}
To obtain a self-assembling molecule in a three dimensional shape, three copies of the antimicrobial sequence were covalently bonded to the scaffold described previously and shown in Figure \ref{fig:capzip}. The AM sequence chosen is a variation on the lactoferricin antimicrobial core RRWQWR: two mutations were introduced to favour the assembly of arms belonging to different molecules in an antiparallel fashion. Specifically, given that the original RRWQWTR sequence is found in a $\beta$-sheet (at least in the crystal lattice), the mutations aim at promoting a similar structure. Therefore, the Glutamine residue and the C-terminal Arginine of the lactoferrin motif were replaced with Threonine and Glutammic acid residues to have a self-complementary sequence RRWTWE: the pairing is promoted by the attraction of opposite charges at the ends of the sequence.

%With the virus architecture adopting an n-fold rotational symmetry, where n is usually 3 or 5 or both,9 a triskel conjugate of RRWTWE was generated to give a self-assembling motif with a trilateral symmetry reminiscent of native cage-like subunits (Fig. 1b–e and S1 in ESI†).

\subsection{A viable systems: experimental background and question}
Many experiments have been performed to verify that capzip had the characteristics it was designed for. The set of experimental results obtained on the molecule has been published in Reference \cite{Castelletto2016}, while more recent results extend and consolidate the previous findings.

\begin{figure}
\begin{center}
\textbf{AFM/TEM \ \ \ \ \ \ \ \ \ \ \ \ \ \ cryo-em}
%\subcaption{}\label{fig:exp_capzip_a}
%\subcaption{} \label{fig:exp_capzip_b}
\bigskip

\textbf{fluo hollow capsule \ \ \ \ \ \ \ fluo RNA uptake}
%\subcaption{} \label{fig:exp_capzip_c}
%\subcaption{} \label{fig:exp_capzip_d}
\caption[Experimental results on capzip]{... Reproduced from \cite{Castelletto2016}} \label{fig:exp_capzip}
\end{center}
\end{figure}

\paragraph{Experimental results} First, the assembly ability have been tested: the peptide does not show assembly in pure water (as verified by Dynamic Light Scattering), while in biological buffer (MOPS, 150 mM) at physiological pH 7.0 it forms capsules with dominating size range of 20-200 nm. This is confirmed by images of the capsules obtained with multiple techniques, namely transmission electron microscopy (TEM) (Figure \ref{fig:exp_capzip}), atomic force microscopy (AFM), and cryo-scanning electron microscopy (SEM).
%
The fine structure of these assemblies appears irregular to the resolution power of such techniques. Some insight into the details of the assembly is given by Circular Dichroism (CD) spectrum, which shows a profile characteristic of $\beta$-turns and contains elements of a $\beta$-sheet structure and indole rings, with minima at $\lambda \sim$ 200 nm and 214 nm \ref{fig:exp_capzip}.
%
Complementary evidence about the overall shape of the assembly was provided by the cross-sectional analysis of the assembled capsules by fluorescence microscopy using fluorescein to label capzip. The signal comes from the wall of the capsule only, showing an inner cavity (Figure \ref{fig:exp_capzip}).
%
Finally, small angle X-ray scattering (SAXS) measurements were consistent with compact capsules interfacing with solvent.

The assembly process is monitored also in combination with small interfering RNAs (siRNA): these sequences are a promising tools for RNA interference techniques which aim at inhibiting the expression of specific genes. However, they are easily degradable and thus difficult to deliver to the target cell without an appropriate vehicle.
%
The co-assembly of a 21 base pairs duplex with the peptide shows the formation of structures similar to the ones of peptide-only capsules: CD spectra highlight the helical signal from RNA (in the shape of an A-DNA form) together with the features proper of the peptide.

These co-assembled structures were tested for siRNA uptake in HeLa cells, showing that the presence of the peptide favours the internalisation of the genetic material. This was tested qualitatively via delivery of fluorescent siRNA (Figure \ref{fig:exp_capzip}): internalisation was observed within the first hours from the transfection in localised regions of the cytoplasm, suggesting an endocytic uptake. This distribution was stable over the first five hours of incubation after which the fluorescence signal decayed.

To quantify the level of RNA internalisation, a mRNA knockdown experiment was performed on a HeLa cell line with two housekeeping genes, ACTB ($\beta$-actin, targeted) and GAPDH (reference)\cite{Crombez2009}; and the results were comparable to those of other transfection reagents\cite{Bustin2009}.
%
The silencing of $\beta$-actin mRNA was detected 22 $\pm$ 2 hours after transfection; and its knockdown ``fitness” was expressed relative to cells treated with siRNA alone (background) and normalised against viable cell counts (Figure \ref{fig:exp_capzip}). Cells treated with capzip remained viable after 24 or 48 hour, showing little toxicity.
%
Lipofectamine\textsuperscript{\textregistered}, used as a positive control, showed higher knockdown levels but also lower cell counts, suggesting cytotoxicity \cite{Elsabahy2013}.
%
The experiment was performed at neutral to positive charge ratio close to one (where each siRNA molecule has a -42$e$ charge and capzip a +6$e$ charge), as test experiments performed at higher peptide-to-siRNA ratio showed no improved uptake.

Lastly, the peptide exerts an antimicrobial function: the non-assembled peptide has shown to be effective against both Gram positive and negative bacteria (E. coli, P. aeruginosa and S. aureus), with no haemolytic effects and minimum inhibitory concentrations typical of other antimicrobial agents.
%
Moreover, on Supported Lipid Bilayer (mixed DLPC and DLPG, 3:1 ratio), the capsules create localized pores within minutes, as proven by AFM experiments repeated in time. The pore depth ranges between 1.4 and 2.2 nm, which is well below the radius of the capsule, however it is sufficient to disrupt the structure of the membrane.
%
Finally, to prove the viability of capzip as antimicrobial agent in vivo, it was used to counteract methicillin-resistant S. aureus (MRSA) infection in G. mellonella larvae: the particular bacterial strain used was susceptible to vancomycin, so that the antimicrobial effect of capzip could be compared with such drug. The larvae treated with capzip showed survival rates comparable to those treated with high doses of vancomycin.

%MANCA: microfluidic results

\paragraph{Open questions} Despite the success of the experiments mentioned above, there is much information still to be uncovered on the precise mechanism of action of such peptide. 

Specifically, both the assembly process and the antimicrobial mechanisms contain some unknown: regarding the former, it is important to understand which amino acids or sub-structures allow the pairing of molecules, whether such pairing is specific or not, how reversible it is, and how rigid the final structure is.
%
Regarding the latter, it must be highlighted what molecules in the membrane the peptide binds to, and how this binding affects the full membrane structure. Finally, as there is evidence that the assembled molecule is a more powerful antimicrobial compound than the single molecule, it is interesting to understand whether a) any cooperative action is taking place or b) the enhanced power is due only to the localised higher concentration.

Even if further experiments or future improvements in the techniques already employed might tackle some of the aspects above in a near future, arguably no experimental outcome can provide an atom-by-atom knowledge of the processes of interest in any time soon. Ideally though, one would like to track each of them, i.e. the processes happening in any the environments capzip has been exposed to (physiological solution, supported lipid bilayers, bacterial extracellular matrix, mammal cell membrane and cytoplasm) both in space and time with the finest level of details, and the impossibility of that leaves large gaps in the understanding of the system.


\section{A computational approach to understand capzip}

The gaps mentioned in the characterisation of the systems prompts for new investigations in order to complement the knowledge already provided.

Beside the quest to enrich the fundamental knowledge on self-assembling peptides and antimicrobial ones, the understanding of this very system is crucial for its further development. We outlined already in Section \ref{sec:amp_design} how antimicrobial peptide design can proceed from already viable templates and empirical principles, when first principles are not available. Similar rules hold for designing self-assembling peptidic materials, which have been explored to obtained tailored delivery vehicles (see Section \ref{sec:organic}).
%
Therefore, a full knowledge of the interactions between peptides and of their assembled structures with the membrane, i.e. of the mechanism of its functions, will drive the engineering of new likewise peptides. Such knowledge-driven design would hopefully provide new blocks suitable for a double action as the one capzip performs, and this in a shorter amount of time than a research based on less information or on a trial-and-error procedure of mutations in the chemical composition or in the architecture of the molecule. A few examples of possible knowledge-related improvements include:
\begin{itemize}
\item the knowledge of capzip binding mode to the bacterial membrane might suggest its suitability for a broad range spectrum or a possible tunability of its action against specific pathogens;
\item understanding the molecule-molecule interactions classifies its robustness and the possibility of designing block which disrupts under particular chemical conditions;
\item querying the electrostatic profile of the assembled structure suggests which type of molecules, other than siRNA, could be efficiently co-assembled and thus delivered.
\end{itemize}

In recent years computational techniques are stepping forward to complement experimental knowledge and complete the picture of how biological systems work. For this reason, it seems natural to query such techniques for this system as well. Zooming into the details of the interactions can be performed via a theoretical modelling of the system in time, and thus through the simulation of its evolution, starting from few basic principles and the knowledge of the chemical composition of its parts. The technique this work focusses on is Molecular Dynamics simulations, which aims at reproducing the behaviour of a system of atoms in a semi-classical description using basic physical laws combined with specific knowledge of the system in exam, as it will be described in details in the next chapter.

Thus it is the aim of this thesis to prove that Molecular Dynamics simulations can clarify further details on the assembly mechanisms of capzip and on its interactions with biological membranes in order to gather more information on the system and contribute in the future to the designed of new molecules with enhanced functional capabilities.

%MORE SPECIFIC QUESTIONS

%\paragraph{Bits}

%In the triskel conjugate, RRWTWE is prone to fold as a b-turn and pair into a b-sheet with another arm of another conjugate.7 Triskelions are then
%compact globular morphology of the capsules observed by AFM in solution (Fig. S3†). 

%The specificity of these antibiotics allows generally to selectively target the desire species of bacteria, however it has the drawback that small modification in the target can easily invalidate the effect of the drug. 

%The typical time at which this resilient population appears varies greatly according to the mechanism targeted by the drug, but also and especially by how this is administered, resulting in the modern ways of treating infections a particularly fertile ground to train drug-resistant bacteria. Key reasons of the acceleration of the mechanism are the more extensive - and sometimes unjustified - use of antibiotics, both in the doses used for patients and in the world wide coverage that nowadays can, luckily, been reached. It must not be forgotten that a large portion, if not the majority, of antibiotics do not come from human treatments but actually from agriculture and breeding.
%
%This increased pressure allow for a fast spread of resistance, but the emergency comes from the parallel lack of new discoveries of novel drugs. The few discovered and approve are kept as last resources and used carefully, both to prevent insurgence of resistance as long as possible and for their collateral effects: indeed, the inefficacy of other drugs has pushed pharmaceutical industry to resort to compounds discarded in the first screening because too toxic for the human body.

%The design of new synthetic AMPs can enhance the self-assembly property to improve the antimicrobial activity and to tailor the final structure into an organized geometry suitable as delivery vehicle \cite{Wei2017, King2014, Norn2016, Korendovych2010, Vauthey2002}.
%The design procedure must necessarily be directed by means of both experimental and theoretical modelling techniques \cite{Fjell2011, Maccari2013, Zhang2015}. Indeed, in silico techniques help to evaluate the physico-chemical characteristics of novel compounds and select the sequences suitable for the purpose. Additionally, they can elucidate the mode of actions of the antimicrobial activity which is still unclear for many AMPs, despite the vast body of experiments performed in the field \cite{Nguyen2011, Chan2006, Ulmschneider2017},

%In this context we want to employ computational techniques to investigate the characteristics of a synthetic antimicrobial peptide designed and tested for its self-assembly characteristics. Capzip \cite{Castelletto2016} is a branched peptide which incorporates three copies of a six amino acids AM sequences (one for each branch, SI Figure \ref{fig:SI_capzip_formula}).
%The sequence (RRWTWE) is a mutation of the antimicrobial region RRWQWR of the bovine lactoferricin peptide (PDB entry 1LFC)\cite{Schibli1999}. The mutations have been introduced to create a self-complementary sequence prone to antiparallel $\beta$-sheet formation with other identical sequences belonging to different molecules, via pairing of opposite charges \cite{Castelletto2016}. The three branches enforce a 3-fold rotational symmetry, reminiscent of the architecture of viral subunits, which aims at generating a three dimensional self-assembling motif, as the branches lies onto different planes.

%One of the key advantages of AMPs over pharmaceutical antibiotics is the ability of some of these peptides to also modulate immune responses. For example, in addition to their direct antimicrobial activity, AMPs can protect the host by a range of mechanisms: chemotactic activity, attracting leukocytes; modulation of host-cell responsiveness to TLR ligands; stimulation of angiogenesi enhancement of leukocyte/monocyte activation and differentiation; and modulation of the expression of proinfl ammatory cytokines/chemokines (Figure 1). For example, LL37 has been shown to attract neutrophils, monocytes, mast cells and T cells. Different groups of AMPs appear to have distinct chemotactic activities from each other. 
%Indeed, a common feature across most of them is the targeting of the bacterial membrane, generally in an unspecific way: membranes of different bacteria share often common characteristics, with all their species classifiable in two groups (Gram positive or negative) according to their membrane(s) structure, and thus AMPs are often effective on many representative of one class, if not both.

%Modification of AMPs with Covalent Bonds,  Modification of AMPs by Changing Amino Acid Content, Modification of AMPs by Amidation, Modification of AMPs with Unnatural Amino Acids, Modification of AMPs with Computer-Assisted Methods, New AMP Design by Homology Modeling

%torres: determining structure-activity relationships. site-directed mutagenesis, computationaldesign approaches, synthetic libraries, template-assisted methodologies, and mechanism-based strat-egies. 

%Gandhi2014 for nanocarrier for siRNA/miRNA

%Joo2016 Andersson2016 AMR to AMPs

%approved peptidic drugs Usmani2017 they are 884
%Recent advances cell-penetrating peptide-assisted drug delivery Langel2015
%Designing improved active peptides for therapeutic approaches against infectious diseases Gomes2018 READ

%Molecular mechanisms of membrane targeting antibiotics Epand2016 READ
%BASIS FOR SELECTIVITY OF CATIONIC ANTIMICROBIAL PEPTIDES FOR BACTERIAL VS MAMMALIAN MEMBRANES 2005Glukhov
%The therapeutic applications of antimicrobial peptides (AMPs): a patent review Kang2017
