\documentclass[12pt,a4paper,openright,oneside,oldfontcommands]{memoir}
%%% preamble %%%
%\usepackage[natbibapa]{apacite} % for citations
%\usepackage{cite}
%\usepackage{biblatex}
%\usepackage[dvips]{graphicx}  % this does not work with PDF images
%\usepackage{times}   % this changes the font
%\usepackage[dvips]{color} % original
\setsecnumdepth{subsection} % this numbers also the subsections 
\usepackage[comma,numbers,sort&compress]{natbib}
\usepackage{graphicx}
\usepackage{amsmath}
\usepackage{amsfonts}
\usepackage{type1cm}
\usepackage{lettrine}
\usepackage{mathrsfs}
\usepackage{color}
\usepackage{multirow}
\usepackage{lscape}
\usepackage{upgreek}
\usepackage{longtable}
\newsubfloat{figure}
\usepackage{amssymb}
%\usepackage{bbm}
\usepackage{csquotes}
\usepackage{etoolbox}
\usepackage{nomencl}
\usepackage{makecell}

%%% set margins %%%
%\setlrmarginsandblock{40mm}{20mm}{*}  % left right margins
%\setulmarginsandblock{35mm}{40mm}{*}  % upper lower margins

\setlrmarginsandblock{40mm}{30mm}{*}  % left right margins
\setulmarginsandblock{35mm}{40mm}{*}  % upper lower margins
\setheadfoot{\headheight}{15mm}
\checkandfixthelayout  % this puts the layout into effect

%----
\usepackage{todonotes}
%\reversemarginpar
%----

%%% customise chapter style %%%
\makeatletter 
\makechapterstyle{thesis}{%
	\renewcommand{\chapnamefont}{\normalfont\LARGE\sffamily}
	\renewcommand{\printchaptername}{\raggedleft\chapnamefont \@chapapp}
	\renewcommand{\chapnumfont}{\normalfont\LARGE\sffamily}
	\renewcommand{\chaptitlefont}{\normalfont\HUGE\sffamily}
	\renewcommand{\printchaptertitle}[1]{%
		\hrule \vskip\onelineskip \raggedleft \chaptitlefont ##1}
	\renewcommand{\afterchaptertitle}{\vskip\onelineskip \hrule\vskip \afterchapskip}}
\makeatother

%%% set line spacing %%%
\DisemulatePackage{setspace}
\usepackage{setspace}
\onehalfspacing

%%% define custom commands %%%
\newcommand{\specialcell}[2][c]{%
  \begin{tabular}[#1]{@{}l@{}}#2\end{tabular}}

\newcommand{\boldalpha}{\mbox{\boldmath$\alpha$}}
\newcommand{\boldsigma}{\mbox{\boldmath$\sigma$}}
\newcommand{\boldxi}{\mbox{\boldmath$\xi$}}

\newcommand{\q}[1]{``#1''}
\newcommand{\red}[1]{\textcolor{red}{#1}}
\newcommand{\blue}[1]{\textcolor{blue}{#1}}
\DeclareMathOperator*{\argmin}{argmin}
\DeclareMathOperator*{\argmax}{argmax}

\chapterstyle{thesis}

\setsecheadstyle{\Large\sffamily\raggedright} 
\setsubsecheadstyle{\large\sffamily\raggedright} 
\setsubsubsecheadstyle{\normalsize\sffamily\raggedright}

\raggedbottomsectiontrue

\usepackage[rm,small]{caption}
\setlength {\captionmargin}{20 pt}

%%% define graphics paths %%%
\graphicspath{{./2gp/}{./3nkernels/}{./4covariant_kernels/}{./5model_selection/}{./6m_ffs/}{./7gp_wf/}}


% La commande qui tue (mais qu'on peut probablement perfectionner):
% 3 paramètres: - préfixe des images - suffixe (extension) des images - options d'affichage des images
\makeatletter
\newcommand\image[3]{
  \@tempcnta\number161
  \advance\@tempcnta-\value{page}
  \divide\@tempcnta2
  \IfFileExists{#1\number\@tempcnta .#2}
  {\includegraphics[#3]{#1\number\@tempcnta .#2}}{}}
\makeatother
% Pb: il faut lui dire combien de page le document a...
% (parce que je sais pas encore faire automatiquement)

% Où l'on place les animations dans les coins.

\setlength{\headwidth}{\textwidth} 
%\addtolength{\headwidth}{\marginparsep} 
\addtolength{\headwidth}{1cm}

\copypagestyle{thesis}{ruled}

\makerunningwidth{thesis}{\headwidth}

\makeheadrule{thesis}{\headwidth}{\normalrulethickness}
\makefootrule{thesis}{\headwidth}{\normalrulethickness}{\footruleskip}

\makeheadposition{thesis}{flushright}{flushleft}{flushright}{flushleft}

%\makeevenfoot{thesis}{}{\thepage}{}

\makeoddfoot{thesis}{}{}{
  \setlength\unitlength{1cm}
  \begin{picture}(0,0)
    \put(-3.5,-1.5){
      \image{./flip/gr}{eps}{scale=0.09}
    }
  \end{picture}
}

\makeatletter
\makepsmarks{thesis}{% 
\let\@mkboth\markboth 
\def\chaptermark##1{\markboth{##1}{##1}}% % left & right marks 
\def\sectionmark##1{\markright{% % right mark 
\ifnum \c@secnumdepth>\z@ 
\thesection. % section number 
\fi 
##1}} 
}
\makeatother

\makeevenhead{thesis}% 
{\normalfont\sffamily\thepage}{}{\normalfont\sffamily\leftmark} 
\makeoddhead{thesis}% 
{\normalfont\sffamily\rightmark}{}{\normalfont\sffamily\thepage} 

\makeevenfoot{thesis}{}{}{}


%%% define hyperreferences links %%%
\usepackage[hyphens]{url}
\usepackage[naturalnames, breaklinks]{hyperref}
%\hypersetup{colorlinks=true,breaklinks=true}
%\hypersetup{		% customise hyperreff
%	colorlinks   = true, 
%	urlcolor     = blue, 
%	linkcolor    = blue, 
%	citecolor   = blue }
